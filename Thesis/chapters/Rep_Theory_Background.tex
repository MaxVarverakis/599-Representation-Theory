\chapter{Representation Theory Background}
\label{ch:rep_background}

\begin{definition}[Representations of a Group]
    If there is a homomorphism from a group $G$ to a group of operators $U(G)$ on a linear vector space $V$, we say that $U(G)$ forms a \textit{representation} of $G$ with dimension $\dim V$.
\end{definition}

The representation is a map
\begin{equation}
    g\in G\xrightarrow{U} U(g)
\end{equation}
in which $U(g)$ is an operator on the vector space $V$. For a set of basis vectors $\{\hat{e_i},i=1,2,\dots,n\}$, we can realize each operator $U(g)$ as an $n\times n$ matrix $D(g)$.
\begin{equation}
    U(g)\ket{e_i} = \sum_{j=1}^n \ket{e_j}{{D(g)}^j}_i = \ket{e_j}{{D(g)}^j}_i,
\end{equation}
where the first index $j$ is the row index and the second index $i$ is the column index. We use the Einstein summation convention, so repeated indices are summed over. Note that the operator multiplication is defined as
\begin{equation}
    U(g_1)U(g_2) = U(g_1g_2),
\end{equation}
which satisfies the group multiplication rules.

\begin{definition}
    If the homomorphism defining the representation is an isomorphism, then the representation is \textit{faithful}. Otherwise, it is \textit{degenerate}.
\end{definition}

\begin{example}
    Let $G$ be the group of continuous rotations in the $xy$-plane about the origin. We can write $G = \{R(\phi),0\leq\phi\leq2\pi\}$ with group operation $R(\phi_1)R(\phi_2) = R(\phi_1+\phi_2)$. Consider the 2-dimensional Euclidean vector space $V_2$. Then we define a representation of $G$ on $V_2$ by the familiar rotation operation
    \begin{align}
        \hat{e}_1' &= U(\phi)\hat{e}_1 = \hat{e}_1\cdot\cos\phi + \hat{e}_2\cdot\sin\phi\\
        \hat{e}_2' &= U(\phi)\hat{e}_2 = -\hat{e}_1\cdot\sin\phi + \hat{e}_2\cdot\cos\phi.
    \end{align}
This gives us the matrix representation
\begin{equation}
    D(\phi) = \begin{pmatrix}
        \cos\phi & -\sin\phi\\
        \sin\phi & \cos\phi
    \end{pmatrix}.
\end{equation}
To further illustrate this representation, if we consider an arbitrary vector $\hat{e_i}x^i=\vec{x}\in V_2$, then we have
\begin{equation}
    \vec{x}' = U(\phi)\vec{x} = \hat{e}_j{x'}^j,
\end{equation}
where ${x'}^j = {{D(\phi)}^j}_i x^i$.
\end{example}

\begin{definition}[Equivalence of Representations]
    For a group $G$, two representations are \textit{equivalent} if they are related by a similarity transformation. Equivalent representations form an equivalence class.
\end{definition}

To determine whether two representations belong to the same equivalence class, we define
\begin{definition}[Characters of a Representation]
    The \textit{character} $\chi(g)$ of an element $g\in G$ in a representation $U(g)$ is defined as $\chi(g) = \text{Tr}~D(g)$.
\end{definition}
Since trace is independent of basis, the character serves as a class label.

Vector space representations of a group have familiar substructures, which are useful in constructing representations of the group.
\begin{definition}[Invariant Subspace]
    Let $U(G)$ be a representation of $G$ on a vector space $V$, and $W$ a subspace of $V$ such that $U(g)\ket{x}\in W$ for all $\vec{x}\in W$ and $g\in G$. Then $W$ is an \textit{invariant subspace} of $V$ with respect to $U(G)$. An invariant subspace is \textit{minimal} or \textit{proper} if it does not contain any non-trivial invariant subspace with respect to $U(G)$.
\end{definition}

The identification of invariant subspaces on vector space representations leads to the following distinction of the representations.
\begin{definition}[Irreducible Representation]
    A representation $U(G)$ on $V$ is \textit{irreducible} if there is no non-trivial invariant subspace in $V$ with respect to $U(G)$. Otherwise, it is \textit{reducible}. If $U(G)$ is reducible and its orthogonal complement to the invariant subspace is also invariant with respect to $U(G)$, then the representation is \textit{fully reducible}.
    
\end{definition}

\begin{example}
    Under the group of 2-dimensional rotations, consider the 1-dimensional subspace spanned by $\ehat_1$. This subspace is not invariant under 2-dimensional rotations, because a rotation of $\ehat_1$ by $\pi/2$ results in the vector $\ehat_2$ that is clearly not in the subspace spanned by $\ehat_1$. A similar argument shows that the subspace spanned by $\ehat_2$ is not invariant under 2-dimensional rotations.

    However, consider the linear combination of basis vectors
    \begin{equation}
        \ehat_\pm = \frac{1}{\sqrt{2}}\left( \mp\ehat_1 + i\ehat_2 \right),
    \end{equation}
    where $i = \sqrt{-1}$. Then a rotation by angle $\phi$, denoted in operator form as $U(\phi)$, acts on $\ehat_\pm$ by
    \begin{align}
        U(\phi)\ket{\ehat_+} &= U(\phi)\frac{1}{\sqrt{2}}(-\ehat_1 + i\ehat_2) \\
        &= \frac{1}{\sqrt{2}}(-U(\phi)\ket{\ehat_1} + iU(\phi)\ket{\ehat_2}) \nonumber \\
        &= \frac{1}{\sqrt{2}}\left( -\ehat_1\cos\phi - \ehat_2\sin\phi -i\ehat_1\sin\phi + i\ehat_2\cos\phi \right) \nonumber \\
        &= \frac{1}{\sqrt{2}}\left( -\ehat_1(\cos\phi+ i\sin\phi) +i\ehat_2(\cos\phi-i\sin\phi) \right) \nonumber \\
        % &= \frac{1}{\sqrt{2}}\big(\ehat_1(-\cos\phi+i\sin\phi) + \ehat_2(i\cos\phi + \sin\phi)\big) \nonumber \\
        % &= \frac{1}{\sqrt{2}}(-\ehat_1 + i\ehat_2)(\cos\phi - i\sin\phi) \nonumber \\
        & \colorbox{red}{\textbf{Not Done}} \nonumber \\
        &= \ehat_+ (\cos\phi - i\sin\phi) \nonumber \\
        &= \ehat_+ e^{-i\phi}, \\
        \textrm{and } U(\phi)\ket{\ehat_-} &= \ehat_- e^{i\phi}.
    \end{align}

\end{example}



The irreducible representation matrices satisfy \colorbox{red}{orthonormality and completeness} relations.\textbf{ Thm. 3.5}?

\begin{example}[Generator of $\sotwo$]
    Consider the rotations of a 2-dimensional Euclidean vector space about the origin. Let $\ehat_1$ and $\ehat_2$ be orthonormal basis vectors of this space. Using geometry, we can determine how a rotation by some angle $\phi$, written in operator form as $R(\phi)$, acts on the basis vectors:
    \begin{align}
        R(\phi)\ehat_1 &= \ehat_1\cos\phi + \ehat_2\sin\phi \label{eq:rot_1}\\
        R(\phi)\ehat_2 &= -\ehat_1\sin\phi + \ehat_2\cos\phi.\label{eq:rot_2}
    \end{align}
    In matrix form, we can write
    \begin{equation}
        R(\phi) = 
        \begin{pmatrix}
            \cos\phi & -\sin\phi \\
            \sin\phi & \cos\phi
        \end{pmatrix}
    \end{equation}
    which allows us to write Eqn.~\ref{eq:rot_1} and Eqn.~\ref{eq:rot_2} in a condensed form
    \begin{equation}
        R(\phi)\ehat_i = \ehat_j{{R(\phi)}^j}_i,
    \end{equation}
    where we are summing over $j=1,2$.

    Now, let $\vec{x}$ be an arbitrary vector in the plane. Then $\vec{x}$ has components $x^i$ in the basis $\{\ehat_i\}$, where $i=1,2$. Equivalently, we can write $\vec{x}=\ehat_i x^i$. Then under rotations, $\vec{x}$ transforms in accordance to the basis vectors
    \begin{align}
        R(\phi)\vec{x} &= R(\phi)\ehat_i x^i \label{eq:rot_vec} \\
        &= \ehat_j{{R(\phi)}^j}_i x^i \nonumber \\
        &= \left( \ehat_1\mat{R(\phi)}{1}{i} + \ehat_2\mat{R(\phi)}{2}{i} \right)x^i \nonumber \\
        &= \left( \ehat_1\cos\phi + \ehat_2\sin\phi \right) x^1 + \left( \ehat_1(-\sin\phi) + \ehat_2\cos\phi \right) x^2 \nonumber \\
        &= \left( x^1\cos\phi - x^2\sin\phi \right)\ehat_1 + \left( x^1\sin\phi + x^2\cos\phi \right)\ehat_2.  \nonumber
    \end{align}

    Observe that $R(\phi)R^\top(\phi) = E$ where $E$ is the identity matrix. This is precisely what defines \textit{orthogonal matrices}. For 2-dimensional vectors in the plane, it is clear that these rotations do not change the length of said vectors. This can be verified by using Eqn.~\ref{eq:rot_vec}:
    \begin{align}
        |R(\phi)\vec{x}|^2 &= |\ehat_j\mat{R(\phi)}{j}{i} x^i|^2 \\
        &= \left|\left( x^1\cos\phi - x^2\sin\phi \right)\ehat_1 + \left( x^1\sin\phi + x^2\cos\phi \right)\ehat_2\right|^2 \nonumber \\
        &= {\left( x^1\cos\phi - x^2\sin\phi \right)}^2 + {\left( x^1\sin\phi + x^2\cos\phi \right)}^2 \nonumber \\
        &= \left( \cos^2\phi + \sin^2\phi \right)x^1 x_1 + \left( \sin^2\phi + \cos^2\phi \right)x^2 x_2 \nonumber \\
        &= x^1 x_1 + x^2 x_2 = |\vec{x}|^2. \nonumber
    \end{align}

    Similarly, notice that for any continuous rotation by angle $\phi$, $\det R(\phi) = \cos^2\phi+\sin^2\phi = 1$. In general, orthogonal matrices have determinant equal to $\pm1$. However, the result of the above determinant of $R(\phi)$ implies that all continuous rotations in the 2-dimensional plane have determinant equal to $+1$. These are the \textit{special orthogonal matrices of rank 2}. This family of matrices is denoted $\sotwo$. Furthermore, there is a one-to-one correspondence with $\sotwo$ matrices and rotations in a plane.

    We define the group of continuous rotations in a plane by letting $R(0) = E$ be the identity element corresponding to no rotation (i.e., a rotation by angle $\phi=0$), and defining the inverse of a rotation as $R^{-1}(\phi) = R(-\phi) = R(2\pi-\phi)$. This group can be called the $\sotwo$ group. Lastly, we define group multiplication as $R(\phi_1)R(\phi_2) = R(\phi_1+\phi_2)$ and note that $R(\phi) = R(\phi\pm2\pi)$, which can be verified geometrically. Thus, group elements of $\sotwo$ can be labelled by the angle of rotation $\phi\in[0,2\pi)$.

    Now we can find a generator of $sotwo$ by considering an infinitesimal rotation, labelled by some infinitesimal angle $d\phi$. Then this is equivalent to the identity plus some small rotation, which we can write as
    \begin{equation}
        R(\textrm{d}\phi) = E - i \textrm{d}\phi J
    \end{equation}
    where the scalar quantity $-i$ is introduced for later convenience and $J$ is some quantity independent of the rotation angle. If we consider the rotation $R(\phi + \textrm{d}\phi)$, then there are two equivalent ways to interpret this rotation
    \begin{align}
        R(\phi + \textrm{d}\phi) &= R(\phi)R(\textrm{d}\phi) = R(\phi)(E - i \textrm{d}\phi J) = R(\phi) - i \textrm{d}\phi R(\phi)J \\
        R(\phi + \textrm{d}\phi) &= R(\phi) + \textrm{d}R(\phi) = R(\phi) + \textrm{d}\phi\frac{\textrm{d}R(\phi)}{\textrm{d}\phi}
    \end{align}
    where the second equation can be thought of as a Taylor expansion of $R(\phi + \textrm{d}\phi)$ about $\phi$. Equating the two expressions for $R(\phi + \textrm{d}\phi)$ yields
    \begin{equation}
        \textrm{d}R(\phi) = -i\textrm{d}\phi R(\phi)J.
    \end{equation}
    Solving this differential equation (with boundary condition $R(0)=E$) provides us with an equation for any group element involving $J$:
    \begin{equation}
        R(\phi) = e^{-i\phi J},
    \end{equation}
    where $J$ is called the \textit{generator} of the group.
\end{example}