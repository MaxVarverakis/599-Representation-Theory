\chapter{An Introduction to Representation Theory}\label{ch:rep_background}

% \section{Introduction}

\textcolor{red}{Add intro chapter}

\textcolor{red}{Intro paragraph to lead into the definitions.}

\begin{definition}[Representation of a group]
    Let $G$ be a group. A \textit{representation} of $G$ is a homomorphism from $G$ to a group of operators on a linear vector space $V$. The dimension of $V$ is the \textit{dimension} or \textit{degree} of the representation.
    % If there is a homomorphism from a group $G$ to a group of operators $X(G)$ on a linear vector space $V$, we say that $X(G)$ forms a \textit{representation} of $G$ with dimension $\dim V$.
\end{definition}

If $X$ is a representation of $G$ on $V$, then $X$ is a map
\begin{equation}
    g\in G\xrightarrow{X} X(g)
\end{equation}
in which $X(g)$ is an operator on the vector space $V$. For a set of basis vectors $\{\e_i,i=1,2,\dots,n\}$, we can realize each operator $X(g)$ as an $n\times n$ matrix:
\begin{equation}
    X(g)\e_i = \sum_{j=1}^n \e_j{{X(g)}^j}_i = \e_j{{X(g)}^j}_i,
\end{equation}
where the first index $j$ is the row index and the second index $i$ is the column index. We use the Einstein summation convention, so repeated indices are summed over. Note that the operator multiplication is defined as
\begin{equation}
    X(g_1)X(g_2) = X(g_1g_2),
\end{equation}
which satisfies the group multiplication rules.

\begin{definition}
    If the homomorphism defining the representation is an isomorphism, then the representation is \textit{faithful}. Otherwise, it is \textit{degenerate}.
\end{definition}

\begin{example}
    The simplest representation of any group $G$ is the \textit{trivial} representation, in which every $g\in G$ is realized by $g\mapsto 1$. This representation is clearly degenerate.
\end{example}

\begin{example}
    Consider the symmetric group $S_n$. The \textit{defining} representation of $S_n$ encodes each $\sigma\in S_n$ by placing a 1 in the $j$-th row and $i$-th column of the matrix $D(\sigma)$ if $\sigma$ sends $i$ to $j$, and 0 otherwise. For example, in $S_3$, the permutation $(23)$ has the matrix representation
    \begin{align*}
        D\big((23)\big) = \begin{bmatrix} 1 & 0 & 0 \\ 0 & 0 & 1 \\ 0 & 1 & 0 \end{bmatrix},
    \end{align*}
    whereas the permutation $(123)$ is realized by the matrix
    \begin{align*}
        D\big((123)\big) = \begin{bmatrix} 0 & 1 & 0 \\ 0 & 0 & 1 \\ 1 & 0 & 0 \end{bmatrix}.
    \end{align*}
\end{example}

The above example involves a finite group. Infinite groups can also have representations, as demonstrated in the following example.
\begin{example}
    Let $G$ be the group of continuous rotations in the $xy$-plane about the origin. We can write $G = \{R(\phi),0\leq\phi\leq2\pi\}$ with group operation $R(\phi_1)R(\phi_2) = R(\phi_1+\phi_2)$. Consider the 2-dimensional Euclidean vector space $V_2$. Then we define a representation of $G$ on $V_2$ by the familiar rotation operation
    \begin{align}
        \e_1' &= X(\phi)\e_1 = \e_1\cdot\cos\phi + \e_2\cdot\sin\phi\\
        \e_2' &= X(\phi)\e_2 = -\e_1\cdot\sin\phi + \e_2\cdot\cos\phi,
    \end{align}
    where $\e_1$ and $\e_2$ are orthonormal basis vectors of $V_2$. This gives us the matrix representation
    \begin{equation}
        X(\phi) = \begin{bmatrix}
            \cos\phi & -\sin\phi\\
            \sin\phi & \cos\phi
        \end{bmatrix}.
    \end{equation}
    To further illuminate this representation, if we consider an arbitrary vector $\e_i x^i=\vec{x}\in V_2$ (with an implicit sum over repeated indices) then we have
    \begin{equation}
        \vec{x}\,' = X(\phi)\vec{x} = \e_j{x'}^j,
    \end{equation}
    where ${x'}^j = {{X(\phi)}^j}_i x^i$.
    \textcolor{red}{Can probably simplify the notation}
\end{example}

\begin{definition}[Equivalence of Representations]
    For a group $G$, two representations are \textit{equivalent} if they are related by a similarity transformation. Equivalent representations form an equivalence class.
\end{definition}

To determine whether two representations belong to the same equivalence class, we define the following.
\begin{definition}[Characters of a Representation]
    The \textit{character} $\chi(g)$ of an element $g\in G$ in a representation $X(g)$ is defined as $\chi(g) = \text{Tr}~D(g)$.
\end{definition}
Since trace is independent of basis, the character serves as a class label.

\section{Irreducibility and Invariant Subspaces}

Vector space representations of a group have familiar substructures, which are useful in constructing representations of the group.
\begin{definition}[Invariant Subspace]
    Let $X(G)$ be a representation of $G$ on a vector space $V$, and $W$ a subspace of $V$ such that $X(g)\vec{x}\in W$ for all $\vec{x}\in W$ and $g\in G$. Then $W$ is an \textit{invariant subspace} of $V$ with respect to $X(G)$. An invariant subspace is \textit{minimal} or \textit{proper} if it does not contain any non-trivial invariant subspace with respect to $X(G)$.
\end{definition}

The identification of invariant subspaces on vector space representations leads to the following distinction of the representations.
\begin{definition}[Irreducible Representation]
    A representation $X(G)$ on $V$ is \textit{irreducible} if there is no non-trivial invariant subspace in $V$ with respect to $X(G)$. Otherwise, it is \textit{reducible}. If $X(G)$ is reducible and its orthogonal complement to the invariant subspace is also invariant with respect to $X(G)$, then the representation is \textit{fully reducible}.
\end{definition}

A reducible representation can be decomposed into a direct sum of irreducible representations. This decomposition is unique up to equivalence.

\begin{example}
    \textcolor{red}{Different example!}
    Under the group of 2-dimensional rotations, consider the 1-dimensional subspace spanned by $\ehat_1$. This subspace is not invariant under 2-dimensional rotations, because a rotation of $\ehat_1$ by $\pi/2$ results in the vector $\ehat_2$ that is clearly not in the subspace spanned by $\ehat_1$. A similar argument shows that the subspace spanned by $\ehat_2$ is not invariant under 2-dimensional rotations.

    % However, if there are eigenvectors of the rotation operator, we then have invariant subspaces formed by the span of the eigenvectors. For example, the eigenvalues of the rotation operator $X(\phi)$ are such that
    % \begin{align*}
    %     \det\left( D(\phi)-\lambda I \right)
    %         &= {\left( \cos\phi-1 \right)}^2 + \sin^2\phi\\
    %         &= 2-2\cos\phi = 0 \\
    %         &\iff \cos\phi = 1 \\
    %         &\iff \phi = 2\pi n, n\in\mathbb{Z}.
    % \end{align*}
    % \begin{align*}
    %     \begin{bmatrix}
    %         \cos\phi & -\sin\phi\\
    %         \sin\phi & \cos\phi
    %     \end{bmatrix} \begin{bmatrix}
    %         x^1\\x^2
    %     \end{bmatrix} &= \begin{bmatrix}
    %         x^1\cos\phi - x^2\sin\phi\\x^1\sin\phi + x^2\cos\phi
    %     \end{bmatrix} = \lambda \begin{bmatrix}
    %         x^1\\x^2
    %     \end{bmatrix},
    % \end{align*}

    % However, consider the linear combination of basis vectors
    % \begin{equation}
    %     \ehat_\pm = \frac{1}{\sqrt{2}}\left( \mp\ehat_1 + i\ehat_2 \right),
    % \end{equation}
    % where $i = \sqrt{-1}$. Then a rotation by angle $\phi$, denoted in operator form as $X(\phi)$, acts on $\ehat_\pm$ by
    % \begin{align}
    %     X(\phi)\ket{\ehat_+} &= X(\phi)\frac{1}{\sqrt{2}}(-\ehat_1 + i\ehat_2) \\
    %     &= \frac{1}{\sqrt{2}}(-X(\phi)\ket{\ehat_1} + iX(\phi)\ket{\ehat_2}) \nonumber \\
    %     &= \frac{1}{\sqrt{2}}\left( -\ehat_1\cos\phi - \ehat_2\sin\phi -i\ehat_1\sin\phi + i\ehat_2\cos\phi \right) \nonumber \\
    %     &= \frac{1}{\sqrt{2}}\left( -\ehat_1(\cos\phi+ i\sin\phi) +i\ehat_2(\cos\phi-i\sin\phi) \right) \nonumber \\
    %     % &= \frac{1}{\sqrt{2}}\big(\ehat_1(-\cos\phi+i\sin\phi) + \ehat_2(i\cos\phi + \sin\phi)\big) \nonumber \\
    %     % &= \frac{1}{\sqrt{2}}(-\ehat_1 + i\ehat_2)(\cos\phi - i\sin\phi) \nonumber \\
    %     & \colorbox{red}{\textbf{Not Done}} \nonumber \\
    %     &= \ehat_+ (\cos\phi - i\sin\phi) \nonumber \\
    %     &= \ehat_+ e^{-i\phi}, \\
    %     \textrm{and } X(\phi)\ket{\ehat_-} &= \ehat_- e^{i\phi}.
    % \end{align}

\end{example}

The irreducible representation matrices satisfy \textcolor{red}{orthonormality and completeness relations.\textbf{ Thm. 3.5}?}

In the following results, we assume that the representation space is finite-dimensional so that the operators can be represented by matrices.

% \textcolor{red}{Schur's Lemmas!}
% If gross, just put outline of proof and say `see citation for more detail'.

\begin{lemma}\label{th:irred_invertible}
    Let $X$ and $Y$ be irreducible representations of some group $G$ such that $TX(g)=Y(g)T$ for all $g\in G$ and some fixed operator $T$. Then $T$ is either zero or invertible.
\end{lemma}
\begin{proof}
    Let $V$ be the representation vector space with which $X$,$Y$, and $T$ operate on. Consider $\vec{v}\in\ker(T)$. Then $TX(g)\vec{v} = Y(g)T\vec{v} = Y(g)\vec{0}=\vec{0}$ for all $g\in G$. Thus, $\ker(T)$ is invariant under $X(G)$. Since $X$ is irreducible, there cannot be any non-trivial invariant subspaces, so either $\ker(T) = \left\{ \vec{0} \right\}$ or $\ker(T) = V$.

    Similarly, the image of $T$ is invariant under $Y(G)$ since $Y(g)T\vec{v} = TX(g)\vec{v}\in\textrm{im}(T)$ for all $g\in G$ and $\vec{v}\in V$. The irreducibility of $Y$ implies that the $Y$-invariant subspace $\textrm{im}(T)\subseteq V$ is either $\left\{ \vec{0} \right\}$ or $V$.

    By the rank-nullity theorem, $T$ is either zero or square and invertible.
\end{proof}

\begin{lemma}[Schur's Lemma]\label{th:schur}
    Let $X$ be an irreducible representation of a group $G$ defined on a vector space $V$. Consider some operator $T$ on $V$ that commutes with all operators $X(g)$ for $g\in G$. Then $T$ is a scalar multiple of the identity matrix, $I$.
\end{lemma}
\begin{proof}
    Let $\lambda$ be an eigenvalue of $T$. Then $\det(T-\lambda I)$ is not invertible. Since $T$ commutes with $X(G)$, it follows that
    \begin{align*}
        \det(T-\lambda I)X(g)=X(g)\det(T-\lambda I),
    \end{align*}
    for all $g\in G$. By \cref{th:irred_invertible}, $\det(T-\lambda I) = 0$ since $X(G)$ is irreducible. Thus, $T=\lambda I$.
\end{proof}

% \begin{theorem}\label{th:irred}
%     Let $G$ be a finite group. The number of irreducible representations of $G$ is equal to the number of conjugacy classes in $G$. Moreover, the degree of each irreducible representation is equal to the size of the corresponding conjugacy class in $G$.
% \end{theorem}
% \begin{proof}
%     \textcolor{red}{DO IT!!!}
% \end{proof}

\begin{corollary}\label{cor:abelian_irred}
    Let $G$ be a finite abelian group. Then the irreducible representations of $G$ are one-dimensional.
\end{corollary}
\begin{proof}
    Suppose $X(G)$ is an irreducible representation of the abelian group $G$. Fix $h\in G$. Since $X$ is a group homomorphism, we have $X(h)X(g)=X(hg)=X(gh)=X(g)X(h)$ for all $g\in G$. By \cref{th:schur}, $X(h) = \lambda_h I$ for some scalar $\lambda_h$. The choice of $h\in G$ was arbitrary, so $X(g) = \lambda_g I$ for all $g\in G$. Thus, the irreducible representations of $G$ are one-dimensional.
    % Since $G$ is abelian, the conjugacy classes of $G$ are the elements of $G$ themselves. By \cref{th:irred}, the number of irreducible representations of $G$ is equal to the number of conjugacy classes of $G$, which is equal to the number of elements of $G$. Furthermore, the degree of each irreducible representation is equal to the size of the corresponding conjugacy class in $G$, which is always 1. Therefore, the irreducible representations of $G$ are one-dimensional.
\end{proof}