\chapter{To-Do List}\label{ch:todo}

\textbf{Potential committee members:}
\begin{itemize}
    \item \textbf{\textcolor{OliveGreen}{Anton Kaul}}
    \item \textbf{Patrick Orson}
    \item \textbf{Eric Brussel}
    \item \textit{Rob Easton}
    % \item Tony Mendes (not likely)
    % \item Emily Hamilton
    % \item Jeffrey Liese
    % \item Ben Richert
    % \item Dana Paquin (not here?)
    
    % \textit{Physics Faculty:}
    % \item Thomas Gutierrez (Quantum info sci.)
    % \item Hilary Jacks (Two-level memory systems)
    % \item Lei Lu (Non-commutative (quantum?) field theory)
    % \item Matt Mewes (Theoretical particle physics)
    % \item Ian Powell (CMT)
    % \item Karl Saunders (Soft CMT)
    % \item Ben Shlaer (Th. particle, string, cosmo.)
\end{itemize}

\begin{center}\rule{.85\textwidth}{0.65pt}\end{center}

\textbf{Questions for grad ed formatting}
\begin{itemize}
    \item Short figure captions.
    
    \begin{center}\rule{.85\textwidth}{0.65pt}\end{center}
    \textbf{Wednesday questions}
    \item \cref{ch:rep_background} Schur proof reference 531/Mendes?
    \item Beefy captions?
    \item Okay to just state $\psi_n^\textbf{r}(\sigma_i)$ matrices?
    \item Show $\psi_n^\textbf{r}(\sigma_i)$ invertible?
    \item Note the Lie algebra vs Lie group distinction in \cref{ch:Phys_applications}?
    \item \textbf{\textcolor{red}{Do?}} Anyon fusion rules. $\tau$ anyon/Fibonacci anyon example. Relate to singlet/triplet states in spin-1/2 system.
    \item Broken inline math?
    \item \textcolor{black!50!white}{Spend some time on \texttt{MATLAB} thing?}
    \begin{center}\rule{.85\textwidth}{0.65pt}\end{center}
    \item Add sigma inverse to 5.1 rot on Hilbert spaces
    \item Discuss fusion rules and non-abelian-ness of them. Can cite paper without going into too much detail about it.

    \begin{center}\rule{.85\textwidth}{0.65pt}\end{center}
    \item[\checkmark] Gauge theory background, QM background.
    \item[\checkmark] Go over \cref{ch:top_background} and see if it needs more examples, maybe push to appendix.
    \item[\checkmark] Do the stuff above listed for \cref{ch:braid_group}.
    \item[\checkmark] Concluding paragraph on first section of \cref{ch:anyons} to lead into the more physics-y stuff.
    \item[\checkmark] Conclusion/future of anyons/braid group in physics.
    \item[\checkmark] Go over \cref{ch:rep_background} with the relevant bullets above in mind.
    \item[X] Note the Lie algebra vs Lie group distinction in \cref{ch:Phys_applications}.
    \item[\checkmark] Intro paragraphs for \cref{ch:Phys_applications} and sections.
    \item[\checkmark] Comment on faithfulness of the Burau representation.
    \item[\checkmark] Introduction ``chapter''.
    \item[\checkmark] Abstract.
    \item[\checkmark] Acknowledgements.
\end{itemize}
