\chapter{Physics Background}\label{ch:physics_background}

\section{Dirac notation}
Bra-ket notation, ``Hilbert space'', inner product, etc.

\section{Ehrenfest's theorem and conserved quantities}\label{sec:conserved_quantities}

    % \begin{itemize}
    %     \item Get specific $\hat{H}$ that commutes with $J$ and $P$. I'm thinking $\hat{H} = \frac{1}{2m}\hat{P}^2 + V(\hat{X})$?
    % \end{itemize}

    Possible reference here~\cite{Hall2013}!
    
    Suppose $G$ is an operator on a quantum Hilbert space of states. The quantity $\braket{G}$ is conserved if
    \begin{align*}
        \frac{d\braket{G}}{dt} = 0.
    \end{align*}
    Recall the time-dependent Schr\"odinger equation
    \begin{align*}
        \hat{H}\psi = i\hbar\frac{d\psi}{dt} \implies \frac{d\psi}{dt} = \frac{1}{i\hbar}\hat{H}\psi.
    \end{align*}

    Then if $G$ is time-independent we have
    \begin{align*}
        \frac{d\braket{G}}{dt} &= \frac{d}{dt}\braket{\psi|G|\psi} \\
        &= \Braket{\frac{d\psi}{dt}\biggl|G\biggr|\psi} + \Braket{\psi\biggl|G\biggr|\frac{d\psi}{dt}} + \cancelto{0}{\Braket{\psi|\frac{\partial G}{\partial t}|\psi}} \\
        &= \Braket{\frac{1}{i\hbar}\hat{H}\psi\biggl|G\biggr|\psi} + \Braket{\psi\biggl|G\biggr|\frac{1}{i\hbar}\hat{H}\psi} \\
        &= \frac{i}{\hbar}\left( \braket{\hat{H}\psi|G|\psi} - \braket{\psi|G|\hat{H}\psi} \right) \\
        &= \frac{i}{\hbar}\left( \braket{\psi|\hat{H}^\dagger G|\psi} - \braket{\psi|G\hat{H}|\psi} \right) \\
        &= \frac{i}{\hbar}\left( \braket{\psi|\hat{H}G|\psi} - \braket{\psi|G\hat{H}|\psi} \right) \textrm{ because }\hat{H}\textrm{ is Hermitian} \\
        &= \frac{i}{\hbar}\bra{\psi}(\hat{H}G - G\hat{H})\ket{\psi} \\
        &= \frac{i}{\hbar}\bra{\psi}[\hat{H},G]\ket{\psi} = 0 \iff [\hat{H},G] = 0.
        % &= \frac{1}{i\hbar}\left( \bra{\psi}\hat{H}J - \bra{\psi}J\hat{H} \right)\ket{\psi} \\
        % &= \frac{1}{i\hbar}\left( \bra{\psi}[\hat{H},J]\right)\ket{\psi} \\
    \end{align*}
    (linear in the second argument). (See Ehrenfest's theorem).

    Thus, if $[\hat{H},G]=0$, it follows that
    \begin{align*}
        \hat{H}G-G\hat{H} = 0
            &\iff \hat{H}G = G\hat{H} \\
            &\iff \iv{G}\hat{H}G = \hat{H}.
            % &\iff G^\dagger\hat{H}G = \hat{H},
            % &\iff \iv{G}\hat{H}G\ket{\psi} = \hat{H}\ket{\psi},
            % &\iff G^\dagger\hat{H}G\ket{\psi} = \hat{H}\ket{\psi}
    \end{align*}
    Therefore, $\iv{G}\hat{H}G$ and $\hat{H}$ share the same eigenvalues (observables), which is only true if $\hat{H}$ is invariant under $G$.
    % Alternatively, it is known that commuting matrices share a common set of eigenvectors, so they are simultaneously diagonalizable in that eigenbasis. Then in this diagonal basis clearly $\iv{G}\hat{H}G = \hat{H}$, which implies that the eigenvalues are the same.
    If $G$ generates a group of transformations, then $\hat{H}$ is invariant under the group of transformations generated by $G$. If $G$ is unitary, this invariance is often expressed as 
    \begin{align*}
        G^\dagger\hat{H}G = \hat{H}.
    \end{align*}
    
    % where the last line ensures that the resulting transformation of $\hat{G}$ by $G$ is Hermitian, and thus corresponds to physical observables. The Hermiticity of $\hat{H}$ is preserved under $G$ if and only if $\hat{H}$ is invariant under the transformations generated by $G$. 
    
    Running the argument in reverse, if $\hat{H}$ is invariant under the transformations generated by $G$, then $[\hat{H},G]=0$, which, by the Ehrenfest theorem, implies that $\braket{G}$ is conserved.