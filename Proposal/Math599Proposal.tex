\documentclass[oneside]{memoir}

\usepackage{enumitem}
\usepackage{cmbright}

\counterwithout{section}{chapter}

\setlength{\parskip}{1em}
\setlength{\parindent}{0pt}

\title{MATH 599 Proposal\\[0.5em]
	Braid Group Representations and Applications\\	 % Title of Proposal
}

\date{\vspace{-5em}}

\begin{document}

\maketitle

%%%%%%%%% Comment out in final Proposal
%
%\vspace{-.75in}
%\begin{center}\texttt{A copy of the LaTeX file is in the Math One Drive}
%\end{center}
%
%%%%%%%%%%%%%%%%%%%%%%%%%%%%%

\section{References}
\begin{itemize}
    \item Tung,~W. (1985). \textit{Group Theory in Physics}. World Scientific Publishing.
    \item Kassel,~C., \& Turaev,~V. (2008). \textit{Braid Groups}. Springer-Verlag.
    \item Rolfsen,~D. (2010). \textit{Tutorial on the braid Groups}. \texttt{DOI:\@10.48550/arXiv.1010.4051}
    \item Gonzalez-Meneses,~J. (2010). \textit{Basic results on braid groups}. \texttt{DOI:\@10.48550/arXiv.1010.0321}
    \item Baez,~J. (1992). \textit{Braids and Quantization}. \texttt{https://math.ucr.edu/home/baez/braids/braids.html}
\end{itemize}

\section{Course Outline}

We aim to investigate the braid group and its representations in the context of mathematical/theoretical physics. The braid group has many profound applications in physics, such as the Yang-Baxter equation and quantum field theory. Since fall quarter will be the first quarter of thesis, we will use this course to explore the braid group and its representations in order to determine a specific area of interest for future research.  Starting where we left off in the spring, we will begin with studying the Lorentz and Poincar\'e groups and their representations.  Then, we will segway into an exploration of the braid group and the relevant representations.  Our goal this fall is to determine a more specific area of interest for thesis in the winter and spring.

\section{Time-line}

Since fall is the exploratory quarter where we will hone in on a specific topic to study for thesis, the second half of the quarter is continent on what we find interesting while gaining a basic understanding of the relevant groups and representations.
\begin{description}[topsep=0pt,itemsep=0ex]
    \item[\parbox{5em}{Weeks 1-2}] Delve into chapter 10 of Tung's book in which we will study the Lorentz and Poincar\'e groups, their representations, and their applications.
    \item[\parbox{5em}{Week 3-5}] Look into first chapter of Kassel and Turaev to begin our study of the braid group.  We will use other references to supplement our study of the braid group and representations.
    \item[\parbox{5em}{Weeks 6-10}] Study braid group representations and their application with respect to the relevant physics with more specificity based off of our interests.  The other resources listed in the References section will serve as a guide towards our exploration of the braid group.
\end{description}


% The following time-line for the course is proposed.  (Chapter references refer to the book Probability with Martingales by D.\,Williams)

% \newpage

\section{Assessment}

Our current plan is to have frequent meetings (once/twice weekly) to discuss what progress we have made on the material under investigation.  In these meetings, we also plan to determine checkpoints for the following week(s) and discuss any questions or concerns that may arise.  Eventually, we will choose specific research questions to investigate in greater depth.

\section{Number of units}

3 Units

\vspace{1in}
% \vfill
\dotfill\\\\
\textbf{Participants}

\parbox{11em}{Max Varverakis} % \dotfill % Student name

\textbf{Supervisor}

\parbox{11em}{Dr. Sean Gasiorek} % \dotfill % Supervisor name

\textbf{Chair of Mathematics}

\parbox{11em}{Dr. Benjamin Richert} % \dotfill % Chair of Math department

\end{document}