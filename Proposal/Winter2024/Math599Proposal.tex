\documentclass[oneside]{memoir}

\usepackage{enumitem}
\usepackage{cmbright}

\counterwithout{section}{chapter}

\setlength{\parskip}{1em}
\setlength{\parindent}{0pt}

\title{MATH 599 Proposal\\[0.5em]
	Braid Group Representations and Applications\\	 % Title of Proposal
}

\date{\vspace{-5em}}

\begin{document}

\maketitle

%%%%%%%%% Comment out in final Proposal
%
%\vspace{-.75in}
%\begin{center}\texttt{A copy of the LaTeX file is in the Math One Drive}
%\end{center}
%
%%%%%%%%%%%%%%%%%%%%%%%%%%%%%

\section{References}
\begin{itemize}
    \item Tung,~W. (1985). \textit{Group Theory in Physics}. World Scientific Publishing.
    \item Kassel,~C., \& Turaev,~V. (2008). \textit{Braid Groups}. Springer-Verlag.
    \item Rolfsen,~D. (2010). \textit{Tutorial on the braid Groups}. \\\texttt{DOI:\@10.48550/arXiv.1010.4051}
    \item Gonzalez-Meneses,~J. (2010). \textit{Basic results on braid groups}. \\\texttt{DOI:\@10.48550/arXiv.1010.0321}
    \item Baez,~J. (1992). \textit{Braids and Quantization}. \\\texttt{https://math.ucr.edu/home/baez/braids/braids.html}
\end{itemize}

\section{Course Outline}

We aim to investigate the braid group and its representations in the context of mathematical/theoretical physics. The braid group has many profound applications in physics, such as the Yang-Baxter equation and quantum field theory. Since winter quarter will be the second quarter of thesis, we will use this course to take a deeper look at the braid group and its representations.  Starting where we left off in the fall, we will continue studying basic braid group theory.  Then, we will begin to focus on the representations of the braid group and their consequences.  Our goal this winter is to fully dive into the representations of the braid groups and understand the emergent physical implications of these representations.

\section{Time-line}

Since winter is the quarter where most of the in-depth studying of the braid group will take place, the beginning of the quarter will focus on understanding the basics of the braid group (multiple interpretations) as well as common representations of this group. The second part of the quarter will be more focused on the relationship between these representations and applications in physics, for instance.
\begin{description}[topsep=0pt,itemsep=0ex]
    \item[\parbox{5em}{Weeks 1-3}] Continue going through Gonzalez-Meneses to understand the basics of the braid group and its representations.
    \item[\parbox{5em}{Weeks 4-6}] Study other resources on braid group representation theory to gain a better understanding of the relationship between the group structure and the representations.
    \item[\parbox{5em}{Weeks 7-10}] Focus on specific applications of the braid group representations to physics. These will be the core concepts studied in this thesis.
\end{description}


% The following time-line for the course is proposed.  (Chapter references refer to the book Probability with Martingales by D.\,Williams)

% \newpage

\section{Assessment}

Our plan is to have frequent meetings (once/twice weekly) to discuss what progress we have made on the material under investigation.  In these meetings, we also plan to determine checkpoints for the following week(s) and discuss any questions or concerns that may arise.

\section{Number of units}

3 Units

\vspace{1in}
% \vfill
\dotfill\\\\
\textbf{Participants}

\parbox{11em}{Max Varverakis} % \dotfill % Student name

\textbf{Supervisor}

\parbox{11em}{Dr. Sean Gasiorek} % \dotfill % Supervisor name

\textbf{Chair of Mathematics}

\parbox{11em}{Dr. Benjamin Richert} % \dotfill % Chair of Math department

\end{document}