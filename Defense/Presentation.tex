\documentclass[compress,aspectratio=169,10pt,usenames,dvipsnames]{beamer}

\mode<presentation>

% \setbeamertemplate{theorems}[numbered]
% \setbeamertemplate{blocks}[rounded][shadow=true]

\setbeamercolor{block title example}{use=structure,fg=CPwhite,bg=CPgreen} % Example title color
\setbeamercolor{block body example}{use=structure,fg=black,bg=CPbeige} % Example body color

\usetheme{CalPoly2023}

% use if you want greyed out steps rather than invisible
% \setbeamercovered{transparent}

%Path relative to the main .tex file 
\graphicspath{ {../FormattedThesis/TikZ} }

\usepackage{graphicx, amsmath, amsthm, amssymb, enumerate, esint, mathtools,multirow, braket}
\usepackage{amsfonts,mathrsfs,float,array}
%\usepackage{xypic}
%\usepackage{tikz-cd}
\usepackage{tikz}
\usetikzlibrary{arrows, shapes, decorations}
%\usepackage{pgfplots}
%\usepackage{rotating}
%\usepackage[shortlabels]{enumitem}
\usepackage{subfigure}
\usepackage{multicol}
%\usepackage{amsmath}
%\usepackage{epsfig}
%\usepackage{graphicx}
%\usepackage[all,knot]{xy}
%\xyoption{arc}
\usepackage{url}
\usepackage{multimedia}
\usepackage[usenames,dvipsnames]{xcolor}

% \newcommand{\twobytwo}[4]{\left(\begin{array}{cc}
% #1 & #2  \\
% #3 & #4 \end{array} \right)}
% \newcommand{\threebythree}[9]{\left( \begin{array}{ccc}
% #1 & #2 & #3 \\
% #4 & #5 & #6 \\
% #7 & #8 & #9 \end{array} \right)}
% \newcommand{\fourbyfour}[9]{\left( \begin{array}{cccc}
% #1 & #2 & \cdots & #3 \\
% #4 & #5 & \cdots & #6\\
% \vdots & \vdots & \cdots  & \vdots \\
% #7 & #8 & \cdots & #9 \end{array} \right)}
% \newcommand{\ip}[2]{\left<  #1, #2 \right> }

\newcommand{\ehat}{\hat{e}}
\newcommand{\mat}[3]{{{#1}^#2}_#3}
\newcommand{\sotwo}{\textrm{SO}{(2)}}
\newcommand{\C}{\mathbb{C}}
\newcommand{\R}{\mathbb{R}}
\newcommand{\Z}{\mathbb{Z}}
\newcommand{\D}{\mathbb{D}}
\newcommand{\Q}{\mathbb{Q}}
\newcommand{\iv}[1]{{ #1 }^{-1}}
\newcommand{\aut}[1]{\textrm{Aut}\!\left( #1 \right)}
\newcommand{\iso}{\simeq}
\newcommand{\niso}{\not\simeq}
\newcommand{\st}{~\big|~}
\newcommand{\tr}[1]{\textrm{tr}\left(#1\right)}
\newcommand{\size}[1]{\left|#1\right|}
\newcommand{\sheet}[2]{\widetilde{#1}_{#2}}
% \newcommand{\ra}{\right\rangle}
% \newcommand{\la}{\left\langle}
% \newcommand{\lra}[1]{\la{#1}\ra}


% \renewcommand{\vec}[1]{\mathbf{#1}} % Uncomment to use bold vectors
\newcommand{\vd}[1]{\dot{\vec{#1}}}

%\theoremstyle{plain}
\newtheorem{prop}[theorem]{Proposition}
\newtheorem{conj}[theorem]{Conjecture}

%https://tex.stackexchange.com/questions/237949/theorem-and-proposition-with-arbitrary-number-in-beamer
%\newcommand{\propnumber}{} % initialize
%\newtheorem*{prop}{Proposition \propnumber}
%\newenvironment{propc}[1]
%  {\renewcommand{\propnumber}{#1}%
%   \begin{shaded}\begin{prop}}
%  {\end{prop}\end{shaded}}

\addtobeamertemplate{navigation symbols}{}{%
    \usebeamerfont{footline}%
    \usebeamercolor[fg]{footline}%
    \hspace{1em}%
    {\footnotesize \insertframenumber/\inserttotalframenumber}
}

\title[Defense]{Representation Theory and its Applications in Physics \\ \;}
\author[Max Varverakis (mvarvera@calpoly.edu)]{Max Varverakis (mvarvera@calpoly.edu)}
\institute[Cal Poly]{}%{California Polytechnic State University} 
\date{June 5, 2024}
\setbeamercolor{section}{use=structure,fg=CPwhite,bg=CPgreen}
\setbeamercolor{section in head/foot}{use=structure,fg=CPwhite,bg=CPgreen}
\setbeamercolor{block title}{use=structure,fg=CPwhite,bg=CPgreen}
\setbeamercolor{block body}{use=structure,fg=black,bg=CPbeige}

\useoutertheme[subsection=false]{smoothbars}
%\setbeamertemplate{footline}[text line]{} % makes the footer EMPTY

\usepackage{beamerouterthemesmoothbars}
% outer themes include default, infolines, miniframes, shadow, sidebar, smoothbars, smoothtree, split, tree

%%%%%%%%%%%%%%THIS WAS A ~XX MINUTE PRESENTATION WITH ALL VISIBLE SLIDES AND NO QUESTIONS DURING THE TALK


\listfiles
\begin{document}

\begin{frame}
\titlepage

\end{frame}

%%%%%%%%%%%%%%%%%%%%%%%%%%%%%%%%%%%%%%%%%%%%%%%%%%%%%%%%%%%%%%%%%%%%

%\begin{frame}
%\tableofcontents
%\end{frame}
\begin{frame}
\textbf{Outline:}
\begin{enumerate}
  % \item Preliminaries
	\item Introduction to Representation Theory
	\item Examples in Physics
	\item The Braid Group
	\item Physical Applications of the Braid Group
\end{enumerate}
\end{frame}

%%%%%%%%%%%%%%%%%%%%%%%%%%%%%%%%%%%%%%%%%%%%%%%%%%%%%%%%%%%%%%%%%%%%%


% \section{Preliminaries}
% \begin{frame}
%   \sectionpage
% \end{frame}

%%%%%%%%%%%%%%%%%%%%%%%%%%%%%%%%%%%%%%%%%%%%%%%%%%%%%%%%%%%%%%%%%%%%%


\section{Introduction to Representation Theory}
\begin{frame}
  \sectionpage
\end{frame}

%%%%%%%%%%%%%%%%%%%%%%%%%%%%%%%%%%%%%%%%%%%%%%%%%%%%%%%%%%%%%%%%%%%%%

\begin{frame}{Definition of a Representation}

\vfill

\begin{definition}
  Let $G$ be a group. A \textit{representation} of $G$ is a homomorphism from $G$ to a group of operators on a linear vector space $V$. The dimension of $V$ is the \textit{dimension} or \textit{degree} of the representation.
\end{definition}

\vfill

\onslide<2->
If $X$ is a representation of $G$ on a vector space $V$, then $X$ is a map
\vspace*{-.5em}
\begin{equation*}
    g\in G\xrightarrow{X} X(g),
\end{equation*}
where $X(g)$ is an operator on the $V$.

\vfill

\onslide<3->
\begin{block}{Remark}
  If $V$ is finite-dimensional with basis $\left\{ \e_1,\e_2,\dots,\e_n \right\}$, then $X$ can be realized as an $n\times n$ matrix.
\end{block}

\vfill

%\begin{center}
%%set up to be used in the 16:9 aspect ratio
%\begin{overlayarea}{11cm}{4cm}
%\begin{center}
%\includegraphics<1>[width=0.73\textwidth]{InitialSystem0}
%\includegraphics<2>[width=0.73\textwidth]{InitialSystem1}
%\includegraphics<3>[width=0.73\textwidth]{InitialSystem2}
%\includegraphics<4>[width=0.73\textwidth]{InitialSystem3}
%\includegraphics<5>[width=0.73\textwidth]{InitialSystem4}
%\end{center}
%\end{overlayarea}
%\end{center}

\end{frame}

%%%%%%%%%%%%%%%%%%%%%%%%%%%%%%%%%%%%%%%%%%%%%%%%%%%%%%%%%%%%%%%%%%%%%

\begin{frame}{Properties of Representations}

\setbeamercovered{transparent}

\begin{block}<2->{Group Multiplication}
  Representations are group morphisms, so they satisfy the group multiplication rule:
  \begin{equation*}
      X(gh) = X(g)X(h), \quad \forall g,h\in G.
  \end{equation*}
\end{block}

\vfill

\begin{block}<3->{Invertibility}
  If $X$ is a representation of $G$, then $X(g)^{-1} = X(g^{-1}),\ \forall g\in G$.
\end{block}

\vfill

\setbeamercovered{invisible}
\onslide<4->
\underline{\bfseries Consequences:}
% \setbeamercovered{transparent}
\begin{enumerate}
  \item<5-> $X(e) = I$, where $e$ is the identity element of the group and $I$ is the identity operator.
  \item<6-> In the matrix presentation of $X$, $X(g)$ is invertible for all $g\in G$.
\end{enumerate} 

\vfill

\end{frame}

%%%%%%%%%%%%%%%%%%%%%%%%%%%%%%%%%%%%%%%%%%%%%%%%%%%%%%%%%%%%%%%%%%%%%

\begin{frame}{Example: The Trivial Representation}

\begin{block}<1->{Trivial Representation of a Group}
  For any group $G$, the trivial representation takes $g\mapsto 1$ for all $g\in G$.
\end{block}

% \vfill

\onslide<2->
\underline{\bfseries Comments:}

\begin{itemize}
  \item<2-> The trivial representation is always one-dimensional.
  \item<3-> For groups with more than one element, the trivial representation is not invertible, so we call it a \textcolor{CPgold}{\textit{degenerate representation}}.
  \item<4-> If a representation is an isomorphism, then it is a \textcolor{CPgold}{\textit{faithful representation}}.
\end{itemize}

\vfill



\end{frame}

%%%%%%%%%%%%%%%%%%%%%%%%%%%%%%%%%%%%%%%%%%%%%%%%%%%%%%%%%%%%%%%%%%%%%

\begin{frame}{Example: A Faithful Representation of $S_n$}
\vfill

\begin{block}<1->{Defining representation of $S_n$}
The defining representation $D$ of $S_n$ encodes the action of the symmetric group on the standard basis of $\R^n$. If a permutation sends $i$ to $j$, then place a 1 the $i$-th column and $j$-th row of the representation matrix.
\end{block}
\vfill

\onslide<2->
E.g., in $S_3$:

\setbeamercovered{transparent}
\begin{align*}
  \onslide<3,5->{D\big((23)\big) = \begin{bmatrix} 1 & 0 & 0 \\ 0 & 0 & 1 \\ 0 & 1 & 0 \end{bmatrix}} \qquad
  \onslide<4->{D\big((123)\big) = \begin{bmatrix} 0 & 0 & 1 \\ 1 & 0 & 0 \\ 0 & 1 & 0 \end{bmatrix}}
\end{align*}
\vfill

\setbeamercovered{invisible}
\begin{itemize}
  \item<5-> The defining representation of $S_n$ is $n$-dimensional.
  \item<6-> This representation is faithful.
\end{itemize}

\vfill

\end{frame}

%%%%%%%%%%%%%%%%%%%%%%%%%%%%%%%%%%%%%%%%%%%%%%%%%%%%%%%%%%%%%%%%%%%%%

\begin{frame}{Example: An Intuitive Representation of 2D Rotations}

\begin{enumerate}
  \item Note that representations work for continuous groups too!
  \item Define rotation group.
  \item Obtain familiar 2D rotation matrix.
  \item Can you think of other ways to represent 2D rotations? What about $e^{i\phi}$ parameterization? How many ways to do this? How many ways are unique? What does it mean to be unique?
\end{enumerate}

\vfill

\begin{block}<2->{Question}
  How do we classify representations of a group?
\end{block}

\vfill

\end{frame}

%%%%%%%%%%%%%%%%%%%%%%%%%%%%%%%%%%%%%%%%%%%%%%%%%%%%%%%%%%%%%%%%%%%%%

\begin{frame}{Character of a Representation}

\begin{enumerate}
  \item Basics of characters.
  \item Similar representations have the same character.
  \item Lead into uniqueness of representations.
\end{enumerate}

\vfill

\begin{block}<2->{Question}
  How many representations does a group have?
\end{block}
\onslide<3->{\textcolor{CPgold}{\textbf{Answer:}} Something to lead into irreducibility.}

\vfill

\end{frame}

%%%%%%%%%%%%%%%%%%%%%%%%%%%%%%%%%%%%%%%%%%%%%%%%%%%%%%%%%%%%%%%%%%%%%

\begin{frame}{Decomposing Representations}

\begin{enumerate}
  \item Define irreducibility.
  \item Relate to invariant subspaces.
  \item Schur's Lemmas?
  \item Note the consequence of abelian groups and one-dimensional representations. Will be useful later\dots
\end{enumerate}

\vfill

\end{frame}

%%%%%%%%%%%%%%%%%%%%%%%%%%%%%%%%%%%%%%%%%%%%%%%%%%%%%%%%%%%%%%%%%%%%%

\begin{frame}{Example: Irreducible Representation of 2D Rotations}

\begin{enumerate}
  \item Back to our rotation group example\dots
  \item Span of $\e_1$ or $\e_2$ not invariant under rotations.
  \item Define $\e_{\pm}$.
  \item Show each is invariant.
  \item Decompose previous representation into direct sum of these two.
\end{enumerate}

\end{frame}

%%%%%%%%%%%%%%%%%%%%%%%%%%%%%%%%%%%%%%%%%%%%%%%%%%%%%%%%%%%%%%%%%%%%%

\begin{frame}{A Note About Irreducibility}

\begin{enumerate}
  \item Irreducible representations are the building blocks of all representations.
  \item Different ways to construct representations from the irreducibles.
  \item The irreducibles offer insight into the structure of the group.
  \item As for applications in physics, we can use irreps to describe the symmetries of physical systems and understand their consequences.
\end{enumerate}

\end{frame}

%%%%%%%%%%%%%%%%%%%%%%%%%%%%%%%%%%%%%%%%%%%%%%%%%%%%%%%%%%%%%%%%%%%%%


\section{Examples in Physics}
\begin{frame}
  \sectionpage
\end{frame}

%%%%%%%%%%%%%%%%%%%%%%%%%%%%%%%%%%%%%%%%%%%%%%%%%%%%%%%%%%%%%%%%%%%%%

\begin{frame}{Preliminaries}

\textbf{Outline:}
\begin{enumerate}
  \item Dirac notation.
  \item Basic quantum mechanics.
  \item Quantum Hilbert space.
  \item The commutator.
\end{enumerate}

\end{frame}

%%%%%%%%%%%%%%%%%%%%%%%%%%%%%%%%%%%%%%%%%%%%%%%%%%%%%%%%%%%%%%%%%%%%%

\begin{frame}{Preliminaries: Dirac Notation}



\end{frame}

%%%%%%%%%%%%%%%%%%%%%%%%%%%%%%%%%%%%%%%%%%%%%%%%%%%%%%%%%%%%%%%%%%%%%

\begin{frame}{Preliminaries: Basic Quantum Mechanics}



\end{frame}

%%%%%%%%%%%%%%%%%%%%%%%%%%%%%%%%%%%%%%%%%%%%%%%%%%%%%%%%%%%%%%%%%%%%%

\begin{frame}{2D Rotations and SO(2)}

\vfill

\begin{enumerate}
  \item The group of 2D rotations is SO(2).
  \item General properties of SO(2).
\end{enumerate}

\vfill

\end{frame}

%%%%%%%%%%%%%%%%%%%%%%%%%%%%%%%%%%%%%%%%%%%%%%%%%%%%%%%%%%%%%%%%%%%%%

\begin{frame}{Infinitesimal Rotations}

\begin{itemize}
    \item Go through the derivation of the generator of SO(2) in an appropriate level of detail.
\end{itemize}

\end{frame}

%%%%%%%%%%%%%%%%%%%%%%%%%%%%%%%%%%%%%%%%%%%%%%%%%%%%%%%%%%%%%%%%%%%%%

\begin{frame}{Recovering the Rotation Matrix from $J$}

\begin{itemize}
    \item Do Taylor expansion thing to get the rotation matrix from $J$ (looks familiar phys majors?)
\end{itemize}

\end{frame}

%%%%%%%%%%%%%%%%%%%%%%%%%%%%%%%%%%%%%%%%%%%%%%%%%%%%%%%%%%%%%%%%%%%%%

\begin{frame}{Irreducible Representations of SO(2)}

\begin{itemize}
  \item Rep generated by $J$ is unitary, $J$ is Hermitian.
  \item SO(2) abelian implies 1D irreps (reference previous thm's).
  \item Construct 1D invariant subspaces, obtain 1D irreps.
  \item Get result about $m\in\Z$ for irrep label.
  \item Mention ortho/completeness relations?
  \item State vector decomposition. Probably don't have time to delve into detailed derivations but would be great to show part of the argument for getting explicit differential form of $J$.
\end{itemize}

\end{frame}

%%%%%%%%%%%%%%%%%%%%%%%%%%%%%%%%%%%%%%%%%%%%%%%%%%%%%%%%%%%%%%%%%%%%%

\begin{frame}{Conservation of Angular Momentum}

\begin{itemize}
  \item Do commutator example with Hamiltonian and $J$.
  \item Discuss implications.
  \item We can do the same thing for translation group which gives us the familiar $\hat{p}$ operator and conservation of linear momentum!
\end{itemize}

\end{frame}

%%%%%%%%%%%%%%%%%%%%%%%%%%%%%%%%%%%%%%%%%%%%%%%%%%%%%%%%%%%%%%%%%%%%%

\begin{frame}{Generalization to 3 Spatial Dimensions}

\begin{itemize}
  \item Show but don't derive $R_{\vec{n}}(\theta)$ decomposition into $\vec{J}$ components.
  \item We have basis from the components of $\vec{J}$.
  \item Ladies and gentlemen, we got SO(3)\dots
  \item $\vec{J}$ component differential forms?
  \item Commutation relations, in some form talk about $J_{\pm},J^2$ and final eigenvalue results.
\end{itemize}

\end{frame}

%%%%%%%%%%%%%%%%%%%%%%%%%%%%%%%%%%%%%%%%%%%%%%%%%%%%%%%%%%%%%%%%%%%%%

\begin{frame}{Connection to Quantum Mechanics}

\begin{itemize}
  \item Discuss connection between generators and quantum operators, eigenvalues and classical observables, discretization (!), etc.
  \item This is the kicker. I will get very excited here probably.
\end{itemize}

\end{frame}

%%%%%%%%%%%%%%%%%%%%%%%%%%%%%%%%%%%%%%%%%%%%%%%%%%%%%%%%%%%%%%%%%%%%%

\begin{frame}{Multi-valued Irreducible Representations and Spinors}
\textcolor{red}{Not sure where to put this\dots}
  
\begin{itemize}
  \item Let's come back to SO(2) for a second\dots
  \item Show $m=1/2$ irreps.
  \item Discuss implications, spinors, etc\dots
\end{itemize}

\end{frame}
  
%%%%%%%%%%%%%%%%%%%%%%%%%%%%%%%%%%%%%%%%%%%%%%%%%%%%%%%%%%%%%%%%%%%%%


\section{The Braid Group}
\begin{frame}
  \sectionpage
\end{frame}

%%%%%%%%%%%%%%%%%%%%%%%%%%%%%%%%%%%%%%%%%%%%%%%%%%%%%%%%%%%%%%%%%%%%%

\begin{frame}{Basic Definitions}

\vfill

\begin{itemize}
  \item Formal definitions.
  \item Physical/intuitive visualization and interpretation.
  \item Standard generators.
  \item Automorphisms of $\pi_1(\D_n)$.
  \item Braid relations in this picture.
  \item 1D Reps.
  \item Burau representation.
  \item Note on faithfulness.
  \item Unitary representation from reduced Burau.
\end{itemize}

\vfill

\end{frame}

%%%%%%%%%%%%%%%%%%%%%%%%%%%%%%%%%%%%%%%%%%%%%%%%%%%%%%%%%%%%%%%%%%%%%


\section{Physical Applications of the Braid Group}
\begin{frame}
  \sectionpage
\end{frame}

%%%%%%%%%%%%%%%%%%%%%%%%%%%%%%%%%%%%%%%%%%%%%%%%%%%%%%%%%%%%%%%%%%%%%

\begin{frame}{Rotations of Quantum Hilbert Space}

\vfill

\begin{itemize}
  \item 1D action on Hilbert space, permuting particles, compare/contrast to bosons/fermions.
  \item Talk about nontrivial braiding effects.
  \item Example of unitary braid rep acting on Hilbert space.
\end{itemize}

\vfill

\end{frame}

%%%%%%%%%%%%%%%%%%%%%%%%%%%%%%%%%%%%%%%%%%%%%%%%%%%%%%%%%%%%%%%%%%%%%

\begin{frame}{Anyons: A Consequence of Braiding}

\begin{itemize}
  \item Introduce anyons.
  \item Discuss how anyons are described by the braid group.
  \item Fusion rules, abelian vs nonabelian anyons.
  \item Non-interacting anyons.
  \item Non-interacting anyons in harmonic potential.
  \item Nontrivial braiding effects anyone?
  \item Applications of anyons! (quantum computing, topological quantum field theory, FQHE, etc.)
\end{itemize}

\end{frame}

%%%%%%%%%%%%%%%%%%%%%%%%%%%%%%%%%%%%%%%%%%%%%%%%%%%%%%%%%%%%%%%%%%%%%

\begin{frame}{Summary/Conclusion}

Acknowledgements, questions, references (?)

\end{frame}

%%%%%%%%%%%%%%%%%%%%%%%%%%%%%%%%%%%%%%%%%%%%%%%%%%%%%%%%%%%%%%%%%%%%%
\end{document}