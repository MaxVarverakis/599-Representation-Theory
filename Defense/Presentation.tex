\documentclass[compress,aspectratio=169,10pt,usenames,dvipsnames]{beamer}
% \documentclass[draft,compress,aspectratio=169,10pt,usenames,dvipsnames]{beamer}

\mode<presentation>

% \setbeamertemplate{theorems}[numbered]
% \setbeamertemplate{blocks}[rounded][shadow=true]

% \setbeamercolor{block title example}{use=structure,fg=CPwhite,bg=CPgreen} % Example title color
% \setbeamercolor{block body example}{use=structure,fg=black,bg=CPbeige} % Example body color

\usetheme{CalPoly2023}

% \setbeamercolor{block body}{use=structure,fg=black,bg=CPbeige!25} % change block body color opacity

\newenvironment<>{pfsketch}{%
  \setbeamercolor{block title}{use=structure,fg=CPwhite,bg=CPblack!90}%
  \setbeamercolor{block body}{use=structure,fg=black,bg=CPblack!5}%
  \begin{block}#1{Proof (sketch)}\let\qed\relax%
}{%
  \end{block}%
}

% \newenvironment<>[1]{alert}{%
%   \setbeamercolor{block title}{use=structure,fg=CPwhite,bg=purple!90}%
%   \setbeamercolor{block body}{use=structure,fg=black,bg=CPblack!5}%
%   \begin{block}#2{#1}%
% }{%
%   \end{block}%
% }

% use if you want greyed out steps rather than invisible
% \setbeamercovered{transparent}

%Path relative to the main .tex file 
\graphicspath{ {../FormattedThesis/TikZ} }

\usepackage{graphicx, amsmath, amsthm, amssymb, enumerate, esint, mathtools,multirow, braket, empheq}
\usepackage{amsfonts,mathrsfs,float,array}
%\usepackage{xypic}
%\usepackage{tikz-cd}
\usepackage{tikz}
\usetikzlibrary{arrows, shapes, decorations}
%\usepackage{pgfplots}
%\usepackage{rotating}
%\usepackage[shortlabels]{enumitem}
\usepackage{subfigure}
\usepackage{multicol}
%\usepackage{amsmath}
%\usepackage{epsfig}
%\usepackage{graphicx}
%\usepackage[all,knot]{xy}
%\xyoption{arc}
\usepackage{url}
\usepackage{multimedia}
\usepackage[usenames,dvipsnames]{xcolor}

% \newcommand{\twobytwo}[4]{\left(\begin{array}{cc}
% #1 & #2  \\
% #3 & #4 \end{array} \right)}
% \newcommand{\threebythree}[9]{\left( \begin{array}{ccc}
% #1 & #2 & #3 \\
% #4 & #5 & #6 \\
% #7 & #8 & #9 \end{array} \right)}
% \newcommand{\fourbyfour}[9]{\left( \begin{array}{cccc}
% #1 & #2 & \cdots & #3 \\
% #4 & #5 & \cdots & #6\\
% \vdots & \vdots & \cdots  & \vdots \\
% #7 & #8 & \cdots & #9 \end{array} \right)}
% \newcommand{\ip}[2]{\left<  #1, #2 \right> }

\newcommand{\ehat}{\hat{\mathbf{e}}}
\newcommand{\e}{\mathbf{e}}
\newcommand{\mat}[3]{{{#1}^#2}_#3}
\newcommand{\sotwo}{\textrm{SO}{(2)}}
\newcommand{\C}{\mathbb{C}}
\newcommand{\R}{\mathbb{R}}
\newcommand{\Z}{\mathbb{Z}}
\newcommand{\D}{\mathbb{D}}
\newcommand{\Q}{\mathbb{Q}}
\newcommand{\iv}[1]{{ #1 }^{-1}}
\newcommand{\aut}[1]{\textrm{Aut}\!\left( #1 \right)}
\newcommand{\iso}{\simeq}
\newcommand{\niso}{\not\simeq}
\newcommand{\st}{~\big|~}
\newcommand{\tr}[1]{\textrm{tr}\left(#1\right)}
\newcommand{\size}[1]{\left|#1\right|}
\newcommand{\sheet}[2]{\widetilde{#1}_{#2}}
% \newcommand{\ra}{\right\rangle}
% \newcommand{\la}{\left\langle}
% \newcommand{\lra}[1]{\la{#1}\ra}


\renewcommand{\vec}[1]{\mathbf{#1}} % Uncomment to use bold vectors
\newcommand{\vd}[1]{\dot{\vec{#1}}}
\newcommand{\vh}[1]{\vec{\hat{#1}}} % bOlDeD hAtS
% \newcommand{\vh}[1]{\hat{\vec{#1}}} % unbolded hats

%\theoremstyle{plain}
\newtheorem{prop}[theorem]{Proposition}
\newtheorem{conj}[theorem]{Conjecture}

%https://tex.stackexchange.com/questions/237949/theorem-and-proposition-with-arbitrary-number-in-beamer
%\newcommand{\propnumber}{} % initialize
%\newtheorem*{prop}{Proposition \propnumber}
%\newenvironment{propc}[1]
%  {\renewcommand{\propnumber}{#1}%
%   \begin{shaded}\begin{prop}}
%  {\end{prop}\end{shaded}}

\addtobeamertemplate{navigation symbols}{}{%
    \usebeamerfont{footline}%
    \usebeamercolor[fg]{footline}%
    \hspace{1em}%
    {\footnotesize \insertframenumber/\inserttotalframenumber}
}

\title[Defense]{Representation Theory and its Applications in Physics \\ \;}
\author[Max Varverakis (mvarvera@calpoly.edu)]{Max Varverakis (mvarvera@calpoly.edu)}
\institute[Cal Poly]{}%{California Polytechnic State University} 
\date{June 5, 2024}
\setbeamercolor{section}{use=structure,fg=CPwhite,bg=CPgreen}
\setbeamercolor{section in head/foot}{use=structure,fg=CPwhite,bg=CPgreen}
\setbeamercolor{block title}{use=structure,fg=CPwhite,bg=CPgreen}
\setbeamercolor{block body}{use=structure,fg=black,bg=CPbeige}

\useoutertheme[subsection=false]{smoothbars}
%\setbeamertemplate{footline}[text line]{} % makes the footer EMPTY

\usepackage{beamerouterthemesmoothbars}
% outer themes include default, infolines, miniframes, shadow, sidebar, smoothbars, smoothtree, split, tree

%%%%%%%%%%%%%%THIS WAS A ~XX MINUTE PRESENTATION WITH ALL VISIBLE SLIDES AND NO QUESTIONS DURING THE TALK


\listfiles
\begin{document}

\begin{frame}
\titlepage

\end{frame}

%%%%%%%%%%%%%%%%%%%%%%%%%%%%%%%%%%%%%%%%%%%%%%%%%%%%%%%%%%%%%%%%%%%%

%\begin{frame}
%\tableofcontents
%\end{frame}
\begin{frame}
\textbf{Outline:}
\begin{enumerate}
  % \item Preliminaries
	\item Introduction to Representation Theory
	\item Examples in Physics
	\item The Braid Group
	\item Physical Applications of the Braid Group
\end{enumerate}
\end{frame}

%%%%%%%%%%%%%%%%%%%%%%%%%%%%%%%%%%%%%%%%%%%%%%%%%%%%%%%%%%%%%%%%%%%%%


% \section{Preliminaries}
% \begin{frame}
%   \sectionpage
% \end{frame}

%%%%%%%%%%%%%%%%%%%%%%%%%%%%%%%%%%%%%%%%%%%%%%%%%%%%%%%%%%%%%%%%%%%%%


\section{Introduction to Representation Theory}
\begin{frame}
  \sectionpage
\end{frame}

%%%%%%%%%%%%%%%%%%%%%%%%%%%%%%%%%%%%%%%%%%%%%%%%%%%%%%%%%%%%%%%%%%%%%

\begin{frame}{Definition of a Representation}

  \vfill

  \begin{definition}
    Let $G$ be a group. A \textit{representation} of $G$ is a homomorphism from $G$ to a group of operators on a linear vector space $V$. The dimension of $V$ is the \textit{dimension} or \textit{degree} of the representation.
  \end{definition}

  \vfill

  \onslide<2->
  If $X$ is a representation of $G$ on a vector space $V$, then $X$ is a map
  \vspace*{-.5em}
  \begin{equation*}
      g\in G\xrightarrow{X} X(g),
  \end{equation*}
  where $X(g)$ is an operator on the $V$.

  \vfill

  \onslide<3->
  \begin{block}{Remark}
    If $V$ is finite-dimensional with basis $\left\{ \e_1,\e_2,\dots,\e_n \right\}$, then $X$ can be realized as an $n\times n$ matrix.
  \end{block}

  \vfill

%\begin{center}
%%set up to be used in the 16:9 aspect ratio
%\begin{overlayarea}{11cm}{4cm}
%\begin{center}
%\includegraphics<1>[width=0.73\textwidth]{InitialSystem0}
%\includegraphics<2>[width=0.73\textwidth]{InitialSystem1}
%\includegraphics<3>[width=0.73\textwidth]{InitialSystem2}
%\includegraphics<4>[width=0.73\textwidth]{InitialSystem3}
%\includegraphics<5>[width=0.73\textwidth]{InitialSystem4}
%\end{center}
%\end{overlayarea}
%\end{center}

\end{frame}

%%%%%%%%%%%%%%%%%%%%%%%%%%%%%%%%%%%%%%%%%%%%%%%%%%%%%%%%%%%%%%%%%%%%%

\begin{frame}{Properties of Representations}

  \setbeamercovered{transparent}

  \begin{block}<2->{Group Multiplication}
    Representations are group morphisms, so they satisfy the group multiplication rule:
    \begin{equation*}
        X(gh) = X(g)X(h), \quad \forall g,h\in G.
    \end{equation*}
  \end{block}

  \vfill

  \begin{block}<3->{Invertibility}
    If $X$ is a representation of $G$, then $X(g)^{-1} = X(g^{-1}),\ \forall g\in G$.
  \end{block}

  \vfill

  \setbeamercovered{invisible}
  \onslide<4->
  \underline{\bfseries Consequences:}
  % \setbeamercovered{transparent}
  \begin{enumerate}
    \item<5-> $X(e) = I$, where $e$ is the identity element of the group and $I$ is the identity operator.
    \item<6-> In the matrix presentation of $X$, $X(g)$ is invertible for all $g\in G$.
  \end{enumerate} 

  \vfill

\end{frame}

%%%%%%%%%%%%%%%%%%%%%%%%%%%%%%%%%%%%%%%%%%%%%%%%%%%%%%%%%%%%%%%%%%%%%

\begin{frame}{Example: The Trivial Representation}

  \begin{block}<1->{Trivial Representation of a Group}
    For any group $G$, the trivial representation takes $g\mapsto 1$ for all $g\in G$.
  \end{block}

  % \vfill

  \onslide<2->
  \underline{\bfseries Comments:}

  \begin{itemize}
    \item<2-> The trivial representation is always one-dimensional.
    \item<3-> For groups with more than one element, the trivial representation is not injective, so we call it a \textcolor{CPgold}{\textit{degenerate representation}}.
    \item<4-> If a representation is injective, then it is a \textcolor{CPgold}{\textit{faithful representation}}.
  \end{itemize}

  \vfill



\end{frame}

%%%%%%%%%%%%%%%%%%%%%%%%%%%%%%%%%%%%%%%%%%%%%%%%%%%%%%%%%%%%%%%%%%%%%

\begin{frame}{Example: A Faithful Representation of $S_n$}

  \vfill

  \begin{block}<1->{Defining representation of $S_n$}
  The defining representation $D$ of $S_n$ encodes the action of the symmetric group on the standard basis of $\R^n$. If a permutation sends $i$ to $j$, then place a 1 the $i$-th column and $j$-th row of the representation matrix.
  \end{block}
  
  \vfill

  \onslide<2->
  E.g., in $S_3$:
  \vspace*{-.75em}
  \setbeamercovered{transparent}
  \begin{align*}
    \onslide<3,5->{D\big((23)\big) = \begin{bmatrix} 1 & 0 & 0 \\ 0 & 0 & 1 \\ 0 & 1 & 0 \end{bmatrix}} \qquad
    \onslide<4->{D\big((123)\big) = \begin{bmatrix} 0 & 0 & 1 \\ 1 & 0 & 0 \\ 0 & 1 & 0 \end{bmatrix}}
  \end{align*}
  
  \vfill

  \setbeamercovered{invisible}
  \begin{itemize}
    \item<5-> The defining representation of $S_n$ is $n$-dimensional.
    \item<6-> This representation is faithful.
  \end{itemize}

  \vfill

\end{frame}

%%%%%%%%%%%%%%%%%%%%%%%%%%%%%%%%%%%%%%%%%%%%%%%%%%%%%%%%%%%%%%%%%%%%%

% \begin{frame}{Next!}

%   \begin{enumerate}
%     \item Note that representations work for continuous groups too!
%     \item Define rotation group.
%     \item Obtain familiar 2D rotation matrix.
%     \item Can you think of other ways to represent 2D rotations? What about $e^{i\phi}$ parameterization? How many ways to do this? How many ways are unique? What does it mean to be unique?
%   \end{enumerate}

%   \vfill

%   \begin{block}{Question}
%     How do we classify representations of a group?
%   \end{block}

%   \vfill

% \end{frame}

%%%%%%%%%%%%%%%%%%%%%%%%%%%%%%%%%%%%%%%%%%%%%%%%%%%%%%%%%%%%%%%%%%%%%

\begin{frame}[t]{Example: Representation of Continuous Rotation Group}

  % \vfill

  \underline{\textit{Representations also work for continuous groups!}}
  
  \vfill

  \onslide<2->
  Let $G = \{R(\phi),0\leq\phi<2\pi\}$ be the group of continuous rotations in the $xy$-plane ($V_2$) about the origin.
  
  \vfill

  \onslide<3->
  \textbf{Group operation:} $R(\phi_1)R(\phi_2) = R(\phi_1+\phi_2) = R(\phi_2)R(\phi_1)$.
  
  \vfill
  
  \onslide<4->
  \textbf{Identity Element:} $R(0) = I$.
  
  \vfill
  
  \onslide<5->
  \textbf{Inverses:} $\iv{R(\phi)} = R(-\phi) = R(2\pi-\phi)$.
  
  \vfill
  
  \onslide<6->
  \textbf{Periodicity Condition:} $R(\phi \pm 2\pi) = R(\phi)$.
  
  \vfill
  
  \onslide<7->
  \textbf{Representation:} Let $X$ be a representation of $R$ on $V_2$ with\footnote<7->{$\e_1$ and $\e_2$ are orthonormal basis vectors of $V_2$.}
  % \begin{onlyenv}<4>
  %   \begin{align*}
  %     X(\phi)\e_1 &= \e_1\cdot\cos\phi + \e_2\cdot\sin\phi \\
  %     X(\phi)\e_2 &= -\e_1\cdot\sin\phi + \e_2\cdot\cos\phi
  %   \end{align*}
  % \end{onlyenv}
  \onslide<7->
  \begin{empheq}[right=\onslide<8>{\empheqrbrace\implies \boxed{X(\phi) = \begin{bmatrix}\cos\phi & -\sin\phi\\\sin\phi & \cos\phi\end{bmatrix}}}]{align*}
    &\begin{aligned}
        X(\phi)\e_1 &= \e_1\cdot\cos\phi + \e_2\cdot\sin\phi \\
        X(\phi)\e_2 &= -\e_1\cdot\sin\phi + \e_2\cdot\cos\phi
    \end{aligned}
  \end{empheq}
  
  \vfill

\end{frame}

%%%%%%%%%%%%%%%%%%%%%%%%%%%%%%%%%%%%%%%%%%%%%%%%%%%%%%%%%%%%%%%%%%%%%

\begin{frame}[t]{Thoughts}

  % \vfill
  \begin{itemize}
    \item<2-> Can you think of other ways to represent 2D rotations?
    \item<3-> What about $e^{i\phi}$ parameterization?
    \item<4-> How many ways can we represent 2D rotations? 
    \item<5->Are certain representations equivalent?
    \item<6-> What does it mean for representations to be equivalent? Unique?
  \end{itemize}
  
  \vfill

  \begin{block}<7->{Question}
    How do we classify representations of a group?
  \end{block}

  \vfill
  
\end{frame}

%%%%%%%%%%%%%%%%%%%%%%%%%%%%%%%%%%%%%%%%%%%%%%%%%%%%%%%%%%%%%%%%%%%%%

% \begin{frame}{Character Outline}

%   \begin{enumerate}
%     \item Basics of characters.
%     \item Similar representations have the same character.
%     \item Lead into uniqueness of representations.
%   \end{enumerate}

%   \vfill

%   \begin{block}{Question}
%     How many representations does a group have?
%   \end{block}
%   % \onslide<3->
%   {\textcolor{CPgold}{\textbf{Answer:}} Something to lead into irreducibility.}

%   \vfill

% \end{frame}

%%%%%%%%%%%%%%%%%%%%%%%%%%%%%%%%%%%%%%%%%%%%%%%%%%%%%%%%%%%%%%%%%%%%%

\begin{frame}[t]{Equivalent Representations}
  
  \begin{definition}<1->
    Two representations are \textit{equivalent} if they are related by a similarity transformation.
  \end{definition}
  
  % \vfill

  \begin{itemize}
    \item<2-> If two representations are equivalent, then their matrix forms have the same \textcolor{CPgold}{\textit{trace}}.
    \item<3-> Equivalent representations form an equivalence class.
  \end{itemize}
  
  \begin{definition}<4->
    The \textcolor{CPgold}{\textit{character}} of a representation is the trace of the representation matrix.
  \end{definition}
  
  \onslide<5->
  E.g., if $g\in G$ and $X$ is a representation of $G$, then the character of $X(g)$ is $\chi(g) = \tr{X(g)}$.
  
  \vfill
  
  \begin{itemize}
    \item<6-> If two representations have the same character for all $g\in G$, then they are equivalent.
    \item<7-> We can use characters to classify representations.
  \end{itemize}

  \vfill

\end{frame}

%%%%%%%%%%%%%%%%%%%%%%%%%%%%%%%%%%%%%%%%%%%%%%%%%%%%%%%%%%%%%%%%%%%%%

% \begin{frame}{Irrep Outline}

%   \begin{enumerate}
%     \item Define irreducibility.
%     \item Relate to invariant subspaces.
%     \item Schur's Lemmas?
%     \item Note the consequence of abelian groups and one-dimensional representations. Will be useful later\dots
%   \end{enumerate}

%   \vfill

% \end{frame}

%%%%%%%%%%%%%%%%%%%%%%%%%%%%%%%%%%%%%%%%%%%%%%%%%%%%%%%%%%%%%%%%%%%%%

\begin{frame}{Decomposing Representations}
  
  \vfill
  
  \begin{definition}<1->
    A representation $X(G)$ on $V$ is \textit{irreducible} if there is no non-trivial invariant subspace\footnote<1->{Invariant subspace $W\subset V$: $X(g)\vec{w}\in W, \, \forall \, \vec{w}\in W$} in $V$ with respect to $X(G)$. Otherwise, $X(G)$ is \textit{reducible}.
  \end{definition}
  
  \vfill

  \onslide<2->
  \underline{\textbf{Comments:}}
  \begin{itemize}
    \item<3-> Irreducible representations are the building blocks of all representations.
    \item<4-> A reducible representation can be decomposed into a direct sum of irreducible representations.
    \item<5-> The decomposition of a representation into irreducibles is unique up to equivalence.
  \end{itemize}
  
  \vfill

\end{frame}

%%%%%%%%%%%%%%%%%%%%%%%%%%%%%%%%%%%%%%%%%%%%%%%%%%%%%%%%%%%%%%%%%%%%%

% \begin{frame}{2D Rotation Irreps Outline}

%   \begin{enumerate}
%     \item Back to our rotation group example\dots
%     \item Span of $\e_1$ or $\e_2$ not invariant under rotations.
%     \item Define $\e_{\pm}$.
%     \item Show each is invariant.
%     \item Decompose previous representation into direct sum of these two.
%   \end{enumerate}

% \end{frame}

%%%%%%%%%%%%%%%%%%%%%%%%%%%%%%%%%%%%%%%%%%%%%%%%%%%%%%%%%%%%%%%%%%%%%

\begin{frame}[t]{Example: Irreducible Representation of 2D Rotations}
  
  % \vfill
  % \onslide<1->
  \textbf{Note:} The subspace spanned by $\e_1$ (or $\e_2$) is \textit{not} invariant under rotations!

  \begin{block}<2->{Invariance of $\e_{\pm}$}
    Let $\e_{\pm} = \frac{1}{\sqrt{2}}\left(\mp\e_1 + i\e_2\right)$. Then, $X(\phi)\e_{\pm} = e^{\pm i\phi}\e_{\pm}$.
  \end{block}
  
  \vfill

  \onslide<3->{
    \underline{\textbf{Decomposition of $X$}}
    
    \vspace{.5em}
    
    The span of each $\e_{\pm}$ is an $X$-invariant subspace of $V_2$. In this basis, we rewrite $X$ as a direct sum of the 1D irreducible representations\footnote<3->{1-dimensional representations are always irreducible!}:
    \begin{equation*}
      X(\phi) = \begin{bmatrix} e^{i\phi} & 0 \\ 0 & e^{-i\phi} \end{bmatrix}.
    \end{equation*}
  }
  
  % Say something about block diagonal matrix form in general?

  \vfill

\end{frame}

%%%%%%%%%%%%%%%%%%%%%%%%%%%%%%%%%%%%%%%%%%%%%%%%%%%%%%%%%%%%%%%%%%%%%

\begin{frame}{Schur's Lemmas (pt. 1)}

  \begin{lemma}<1->
    Let $X:G\to V$ and $Y:G\to W$ be irreducible representations of a group $G$. If there exists a fixed linear transformation $T:V\to W$ such that $TX(g) = Y(g)T$ for all $g\in G$, then $T$ is either the zero map or invertible.
  \end{lemma}

  \begin{pfsketch}<2->
    \begin{enumerate}
      \item<3-> The kernel of $T$ is invariant under $X(G)$.
      \item<4-> The image of $T$ is invariant under $Y(G)$.
      \item<5-> Since $X$ and $Y$ are irreducible, $\ker(T) = \left\{ \vec{0} \right\}$ and $\text{im}(T) = V$ or $\ker(T) = V$ and $\text{im}(T) = \left\{ 0 \right\}$.
      \item<6-> By the rank-nullity theorem, conclude that $T$ is either the zero map or invertible.
    \end{enumerate}
  \end{pfsketch}
  
  \vfill

\end{frame}

%%%%%%%%%%%%%%%%%%%%%%%%%%%%%%%%%%%%%%%%%%%%%%%%%%%%%%%%%%%%%%%%%%%%%

\begin{frame}{Schur's Lemma's (pt. 2)}

  \begin{lemma}
    Let $X$ be an irreducible representation of a group $G$ and $T$ a linear operator that commutes with all $X(g)$ for $g\in G$. Then $T$ is a scalar multiple of the identity operator.
  \end{lemma}

  \begin{pfsketch}<2->
    \begin{enumerate}
      \item<3-> Consider $\lambda$ to be an eigenvalue of $T$.
      \item<4-> Then $T - \lambda I$ is not invertible.
      \item<5-> By assumption, $(T-\lambda I)X(g)=X(g)(T-\lambda I)$ for all $g\in G$.
      \item<6-> By previous lemma, $T-\lambda I = 0 \implies T = \lambda I$.
    \end{enumerate}
  \end{pfsketch}

  \vfill
  
\end{frame}

%%%%%%%%%%%%%%%%%%%%%%%%%%%%%%%%%%%%%%%%%%%%%%%%%%%%%%%%%%%%%%%%%%%%%

\begin{frame}{Consequence of Schur's Lemmas}
  
  \begin{corollary}
    If $G$ is a finite abelian group, then the irreducible representations of $G$ are one-dimensional.
  \end{corollary}

  \begin{pfsketch}<2->
    \begin{enumerate}
      \item<3-> Fix $h\in G$.
      \item<4-> Since $G$ is abelian, $X(h)X(g) = X(g)X(h)$ for all $g\in G$.
      \item<5-> Schur's second lemma implies $X(h)=\lambda_h I$ for some scalar $\lambda_h$.
      \item<6-> The element $h$ was arbitrary, so $X(g) = \lambda_g I$ for all $g\in G$.
      \item<7-> $X(G)$ is equivalent to the representation $g\mapsto \lambda_g$ for all $g\in G$.
      \item<8-> One-dimensional representation are irreducible.
    \end{enumerate}
  \end{pfsketch}

  \vfill

\end{frame}

%%%%%%%%%%%%%%%%%%%%%%%%%%%%%%%%%%%%%%%%%%%%%%%%%%%%%%%%%%%%%%%%%%%%%

\begin{frame}{A Note About Irreducibility}

  \begin{itemize}
    \item<2-> Irreducible representations are the building blocks of all representations.
    \vspace{1em}
    \item<3-> Irreducible representations can be combined/modified to create new representations, such as:
    \begin{itemize}
      \item<4-> Direct sums
      \item<5-> Tensor products
      \item<6-> Complex conjugation\footnote<6->{If the representation matrices have entries in $\C$.}
      \item<7-> Similarity transforms
    \end{itemize}
  \end{itemize}

  \vfill

  \begin{block}<8->{How does this help in physics?}
    Irreducible representations can describe symmetries of physical systems with remarkably fundamental implications.
  \end{block}

  % \onslide<4-> \textbf{Physics perspective:} irreducible representations can describe symmetries of physical systems and have important consequences.

  \vfill
\end{frame}

%%%%%%%%%%%%%%%%%%%%%%%%%%%%%%%%%%%%%%%%%%%%%%%%%%%%%%%%%%%%%%%%%%%%%


\section{Examples in Physics}
\begin{frame}
  \sectionpage
\end{frame}

%%%%%%%%%%%%%%%%%%%%%%%%%%%%%%%%%%%%%%%%%%%%%%%%%%%%%%%%%%%%%%%%%%%%%

\begin{frame}{Preliminaries}

  \textcolor{red}{\textbf{Skip preliminaries?}}

  % \textbf{Outline:}
  % \begin{enumerate}
  %   \item Dirac notation.
  %   \item Basic quantum mechanics.
  %   \item Quantum Hilbert space.
  %   \item The commutator.
  % \end{enumerate}

\end{frame}

%%%%%%%%%%%%%%%%%%%%%%%%%%%%%%%%%%%%%%%%%%%%%%%%%%%%%%%%%%%%%%%%%%%%%

\begin{frame}[t]{Preliminaries: Physics Conventions}
  
  \begin{enumerate}
    \item<2-> The quantum state of a system is described by a vector in a complex Hilbert space.
    \item<3-> The corresponding vectors are often called \textit{\textcolor{CPgold}{state vectors}}.
    \item<4-> The inner product defined on the Hilbert space is linear in the second argument:
    \setbeamercovered{transparent}
    \begin{equation*}
      \onslide<5,7->{(1)\ \ \langle \phi,\lambda\psi\rangle=\lambda\langle \phi,\psi\rangle} \onslide<6->{\qquad (2)\ \ \langle \alpha\phi,\psi\rangle=\overline{\alpha}\langle \phi,\psi\rangle}
    \end{equation*}
    \setbeamercovered{invisible}
    \item<7-> The \textcolor{CPgold}{\textit{Hermitian conjugate}} or \textcolor{CPgold}{\textit{adjoint}} of an operator $A$ is denoted $A^{\dagger}$, and is thought of as complex conjugation and transposition in matrix form.
    \item<8-> Operators that are self-adjoint are called \textcolor{CPgold}{\textit{Hermitian}}.
  \end{enumerate}
  
  \vfill

\end{frame}

%%%%%%%%%%%%%%%%%%%%%%%%%%%%%%%%%%%%%%%%%%%%%%%%%%%%%%%%%%%%%%%%%%%%%

\begin{frame}{Preliminaries: Dirac notation}
  
  \begin{itemize}
    \item<2-> \textit{Dirac} or \textit{bra-ket} notation is a convenient way to represent vectors and operators in quantum mechanics.
    \item<3-> A \textcolor{CPgold}{\textit{ket}} is a column (state) vector, denoted $\ket{\psi}$.
    \item<4-> A \textcolor{CPgold}{\textit{bra}} is a row vector, $\bra{\psi}$. This can be thought of as a linear functional on the relevant Hilbert space:
    \begin{equation*}
      \bra{\phi}(\psi) = \langle \phi,\psi\rangle.
    \end{equation*}
    \item<5-> \textbf{Inner product:} $\braket{\phi|\psi}$
    \item<6-> \textbf{Outer product:} $\ket{\phi}\bra{\psi}$
    \item<7-> The action of an operator $A$ on a vector $\ket{\psi}$ is written as $\ket{A\psi} = A\ket{\psi}$.
    \item<8-> Equivalent ways to write the same thing:
    \begin{align*}
      \braket{A^\dagger\phi|\psi} = \bra{\phi}A\ket{\psi} = \braket{\phi|A\psi}.
    \end{align*}
  \end{itemize}
  
  \vfill
  % \textit{``It's like playing with LEGOs.''}
  
\end{frame}

%%%%%%%%%%%%%%%%%%%%%%%%%%%%%%%%%%%%%%%%%%%%%%%%%%%%%%%%%%%%%%%%%%%%%

\begin{frame}{Orthonormality, Completeness, and Wavefunctions}
  
  \begin{definition}<1->
    Let $\left\{ \ket{1},\ket{2},\ket{3},\dots \right\}$ be an orthonormal basis for some quantum Hilbert space. In the context of physics, the \textcolor{CPgold}{orthonormality} and \textcolor{CPgold}{completeness} relations of the basis vectors allow any state vector $\ket{\psi}$ to be written as a linear combination of the basis vectors:
    \begin{align*}
      \ket{\psi} = \left( \sum_n \ket{n}\bra{n} \right)\ket{\psi} = \sum_n \ket{n}\braket{n|\psi},
    \end{align*}
    where $\sum_n \ket{n}\bra{n}$ is the identity operator.
  \end{definition}
  \onslide<2->\textit{This is just a fancy change of basis!}

  \begin{definition}<3->
    For a continuous basis labelled by $\ket{x}$ where $x$ is a continuous parameter, the \textcolor{CPgold}{\textit{wavefunction}} $\psi(x)$ is the projection: $\braket{x|\psi} = \psi(x)$.
  \end{definition}
  
  \vfill
  
\end{frame}

%%%%%%%%%%%%%%%%%%%%%%%%%%%%%%%%%%%%%%%%%%%%%%%%%%%%%%%%%%%%%%%%%%%%%

\begin{frame}{Preliminaries: Basic Quantum Mechanics}

  \begin{itemize}
    \item Talk about probabilities and whatnot? Eigenvalues = observables? Or just mention when connecting later stuff to physics?
  \end{itemize}

\end{frame}

%%%%%%%%%%%%%%%%%%%%%%%%%%%%%%%%%%%%%%%%%%%%%%%%%%%%%%%%%%%%%%%%%%%%%

\begin{frame}{Properties of 2D Rotations}

  % \vfill
  Let $R$ denote the familiar rotation matrix representation from before.
  
  \begin{block}<2->{Definition}
    An \textcolor{CPgold}{\textit{orthogonal matrix}} $O$ satisfies $O^{\top} = \iv{O}$.
  \end{block}
  
  \vfill
  
  \onslide<3->{
    \textbf{Rotation matrices are orthogonal:}
    \begin{equation*}
      R(\phi)R^\top(\phi) = \begin{bmatrix}\cos\phi & -\sin\phi\\\sin\phi & \cos\phi\end{bmatrix}\begin{bmatrix}\cos\phi & \sin\phi\\-\sin\phi & \cos\phi\end{bmatrix} = \begin{bmatrix}
        1 & 0 \\ 0 & 1
      \end{bmatrix} = I
    \end{equation*}
  }

  \onslide<4->{
    \textbf{Rotations preserve vector lengths:}
    % \begin{align*}
    %   |R(\phi)\vec{x}|^2
    %     &= \left|\left( x_1\cos\phi - x_2\sin\phi \right)\e_1 + \left( x_1\sin\phi + x_2\cos\phi \right)\e_2\right|^2 \\
    %     &= {\left( x_1\cos\phi - x_2\sin\phi \right)}^2 + {\left( x_1\sin\phi + x_2\cos\phi \right)}^2 \\
    %     &= \left( \cos^2\phi + \sin^2\phi \right)x_1^2 + \left( \sin^2\phi + \cos^2\phi \right)x_2^2 \\
    %     &= x_1^2 + x_2^2 = |\vec{x}|^2.
    % \end{align*}
    \begin{align*}
      R(\phi)\vec{x} &= \begin{bmatrix}\cos\phi & -\sin\phi\\\sin\phi & \cos\phi\end{bmatrix}\begin{bmatrix}
        x_1 \\ x_2
      \end{bmatrix} = \begin{bmatrix}
        x_1\cos\phi - x_2\sin\phi \\ x_1\sin\phi + x_2\cos\phi
      \end{bmatrix} \implies \size{R(\phi)\vec{x}}^2 = \size{\vec{x}}^2.
    \end{align*}
  }
  \onslide<5->{
    This \textcolor{CPgold}{special} property is summarized by noting $\det R(\phi) = 1$ for all $\phi\in[0,2\pi)$.
  }

  \vfill

\end{frame}

%%%%%%%%%%%%%%%%%%%%%%%%%%%%%%%%%%%%%%%%%%%%%%%%%%%%%%%%%%%%%%%%%%%%%

\begin{frame}[t]{The SO(2) Group}

  \begin{block}<1->{Definition}
    The \textcolor{CPgold}{\textit{special orthogonal group}} in two dimensions, denoted SO(2), is the group of all $2\times 2$ orthogonal matrices with determinant equal to  $+1$.\footnote<1->{For all intents and purposes, SO(2) is $R$ from before.}
  \end{block}

  \vfill

  \onslide<2->\underline{\textbf{Properties of SO(2):}}
  \begin{itemize}
    \item<3-> The \textit{periodicity condition} $R(\phi+2\pi) = R(\phi)$ is satisfied.
    \item<4-> The \textit{identity element} is $R(0) = I$.
    \item<5-> SO(2) is \textit{abelian}: $R(\phi_1)R(\phi_2) = R(\phi_1+\phi_2) = R(\phi_2)R(\phi_1)$.
    \item<6-> SO(2) is \textit{reducible} (earlier example with $\e_{\pm}$).
  \end{itemize}
  
  \vfill
  
  % \begin{block}<7->{Question}
  %   Why are we talking about rotations again?
  % \end{block}
  % \onslide<7->\textcolor{CPgold}{\textbf{Answer:}} \textit{The irreducible representations of SO(2) will provide powerful insights in quantum mechanics. But first, a slight detour is in order\dots}
  
  % \vfill

\end{frame}

%%%%%%%%%%%%%%%%%%%%%%%%%%%%%%%%%%%%%%%%%%%%%%%%%%%%%%%%%%%%%%%%%%%%%

\begin{frame}[t]{Infinitesimal Rotations}

  \begin{itemize}
    \item<1-> Consider an \textit{infinitesimal rotation} labelled by some infinitesimal angle $d\phi$.
    \item<2-> This is equivalent to the identity plus some small rotation, which can be written as\footnote<2->{The constant $-i$ is introduced for later convenience, and $J$ is a quantity independent of $\phi$.}
    \begin{equation*}
        R(d\phi) = I - i \, d\phi J
    \end{equation*}
    \item<3-> There are two ways to interpret $R(\phi_d\phi)$:
    \begin{align*}
      R(\phi + d\phi) &= R(\phi)R(d\phi) = R(\phi)(I - i d\phi J) = R(\phi) - i d\phi R(\phi)J, \\
      R(\phi + d\phi) &= R(\phi) + d\phi\frac{dR(\phi)}{d\phi}
    \end{align*}
    \item<4-> Equating the two expressions gives the differential equation $dR(\phi) = -id\phi R(\phi)J$.
    \item <5-> With $R(0)=I$ boundary condition: $\boxed{R(\phi) = e^{-i\phi J}}$.
    \item<6-> We call $J$ the \textcolor{CPgold}{\textit{generator}} of SO(2) rotations.
  \end{itemize}

  \vfill

\end{frame}

%%%%%%%%%%%%%%%%%%%%%%%%%%%%%%%%%%%%%%%%%%%%%%%%%%%%%%%%%%%%%%%%%%%%%

\begin{frame}{Recovering the Rotation Matrix from $J$}

  To first order in $d\phi$: $R(d\phi) = \begin{bmatrix}
      1 & -d\phi \\
      d\phi & 1
  \end{bmatrix}$
  % \begin{align*}
  %     R(d\phi) &= \begin{bmatrix}
  %         1 & -d\phi \\
  %         d\phi & 1
  %     \end{bmatrix}.
  % \end{align*}
  
  \vfill

  \onslide<2->{
    From before: $I - i d\phi J = \begin{bmatrix}
      1 & -d\phi \\
      d\phi & 1
    \end{bmatrix} \implies J = \begin{bmatrix}
        0 & -i \\
        i & 0
    \end{bmatrix}\implies J^2=I$
    % \begin{align*}
    %   I - i d\phi J &= \begin{bmatrix}
    %     1 & -d\phi \\
    %     d\phi & 1
    %   \end{bmatrix} \implies J = \begin{bmatrix}
    %       0 & -i \\
    %       i & 0
    %   \end{bmatrix}
    % \end{align*}
  }

  \vfill

  \onslide<3->{
    Taylor expand:\vspace{-.5em}
    \begin{align*}
      R(\phi) = e^{-iJ\phi} &= I - iJ\phi - I \frac{\phi^2}{2!} + iJ\frac{\phi^3}{3!} + \cdots \\
      \onslide<4->{&= I\left( \sum_{n=0}^{\infty} {(-1)}^n \frac{\phi^{2n}}{(2n)!} \right) - iJ\left( \sum_{n=0}^{\infty} {(-1)}^n \frac{\phi^{2n+1}}{(2n+1)!} \right) \\}
      \onslide<5->{&= I\cos\phi - iJ\sin\phi \\}
      \onslide<6->{&= \begin{bmatrix}
          \cos\phi & -\sin\phi \\
          \sin\phi & \cos\phi
      \end{bmatrix}.}
    \end{align*}
  }

  \vfill

\end{frame}

%%%%%%%%%%%%%%%%%%%%%%%%%%%%%%%%%%%%%%%%%%%%%%%%%%%%%%%%%%%%%%%%%%%%%

% \begin{frame}{SO(2) Irreps Outline}

%   \begin{itemize}
%     \item Rep generated by $J$ is unitary, $J$ is Hermitian.
%     \item SO(2) abelian implies 1D irreps (reference previous thm's).
%     \item Construct 1D invariant subspaces, obtain 1D irreps.
%     \item Get result about $m\in\Z$ for irrep label.
%     \item Mention ortho/completeness relations?
%     \item State vector decomposition. Probably don't have time to delve into detailed derivations but would be great to show part of the argument for getting explicit differential form of $J$.
%   \end{itemize}

% \end{frame}

%%%%%%%%%%%%%%%%%%%%%%%%%%%%%%%%%%%%%%%%%%%%%%%%%%%%%%%%%%%%%%%%%%%%%

\begin{frame}{Irreducible Representations of SO(2)}
  
  \textbf{Process to obtaining irreducibles:}
  \begin{enumerate}
    \item<1-> Let $U$ be any representation of SO(2).
    \item<2-> Same argument as before: $U(\phi) = e^{-iJ\phi}$, where $J$ is not necessarily the same as before.
  %   \item<3-> Rotations preserve vector lengths\footnote<3->{$\braket{a|b}$ is a fancy inner product that's linear in the \textit{second} argument.}:
  %   \begin{align*}
  %     \size{\vec{a}}^2 = \size{U(\phi)\vec{a}}^2
  %     &\iff \braket{a|a} = \braket{U(\phi)a|U(\phi)a} = \braket{a|{U(\phi)}^\dagger U(\phi)|a} \\
  %     &\iff U(\phi)^\dagger U(\phi) = I \\
  %     &\iff e^{iJ^\dagger\phi}e^{-iJ\phi} = e^{-i(J-J^\dagger)\phi} = 1 \\
  %     &\iff J = J^\dagger \quad (J\text{ is \textit{Hermitian}!})
  %   \end{align*}
  \item<3-> SO(2) is abelian: Schur's Lemmas $\implies$ all irreducible representations are 1D.
  \item<4-> Each invariant subspace is spanned by an eigenvector of $J$:
  \begin{align*}
    J\ket{m} &= m\ket{m}, \\
    U(\phi)\ket{m} &= e^{-iJ\phi}\ket{m} = e^{-im\phi}\ket{m}.
  \end{align*}
  \item<5-> Periodicity of SO(2) $\implies e^{-i2\pi m} = 1 \implies m\in\Z$.
  \end{enumerate}

  \begin{theorem}<6->
    The \textit{single-valued irreducible representations of SO(2)} are defined as
  \begin{equation*}
      U^m(\phi) = e^{-im\phi}, \ \forall\, m\in\Z.
  \end{equation*}
  \end{theorem}

  \vfill
  
\end{frame}

%%%%%%%%%%%%%%%%%%%%%%%%%%%%%%%%%%%%%%%%%%%%%%%%%%%%%%%%%%%%%%%%%%%%%

\begin{frame}[t]{Generalization to 3 Spatial Dimensions}

  \begin{itemize}
    \item<1-> In 3 spatial dimensions, every rotation can be thought of a rotation in a plane with some perpendicular axis of rotation $\vec{n}$: $R_{\vec{n}}(\theta)$.
    \item<2-> Rotations in a plane are isomorphic to SO(2): $R_{\vec{n}}(\theta) = e^{-i\theta J_{\vec{n}}}$ for some generator $J_{\vec{n}}$.
    \item<3-> The standard generators along each axis $\left\{ J_x,J_y,J_z \right\}$ form a basis for all rotation generators: $J_{\vec{n}} = n_x J_x + n_y J_y + n_z J_z$.
  \end{itemize}
  
  \vfill
  
  \onslide<4->{
    \textbf{Consequence:} Any rotation in Euclidean 3-space can be written in terms of the generators:
    \begin{align*}
      R_{\vec{n}}(\theta) = e^{-i\theta J_{\vec{n}}} = e^{-i\theta(n_x J_x + n_y J_y + n_z J_z)} = e^{-i\theta\,\vec{n}\cdot\vec{J}}.
    \end{align*}
  }

  \vspace{-1.5em}
  
  \begin{definition}<5->
    The \textcolor{CPgold}{\textit{special orthogonal group}} in three dimensions, denoted SO(3), is the group of all $3\times 3$ orthogonal matrices with determinant equal to $+1$. SO(3) rotations are generated by the components of the Hermitian generator $\vec{J} = [J_x,J_y,J_z]^\top$.
  \end{definition}

  % \begin{itemize}
  %   \item Show but don't derive $R_{\vec{n}}(\theta)$ decomposition into $\vec{J}$ components.
  %   \item We have basis from the components of $\vec{J}$.
  %   \item Ladies and gentlemen, we got SO(3)\dots
  %   \item $\vec{J}$ component differential forms?
  %   \item Commutation relations, in some form talk about $J_{\pm},J^2$ and final eigenvalue results.
  % \end{itemize}

  \vfill

\end{frame}

%%%%%%%%%%%%%%%%%%%%%%%%%%%%%%%%%%%%%%%%%%%%%%%%%%%%%%%%%%%%%%%%%%%%%

\begin{frame}[t]{Connection to Quantum Mechanics}

  Using a similar process to generate SO(3) invariant subspaces that correspond to irreducible representations, we summarize the results in a theorem:
  \begin{theorem}<2->
    The irreducible representations of SO(3) are labeled by $j = 0,\frac{1}{2},1,\frac{3}{2},2,\dots$, and the $2j+1$ eigenvectors spanning an invariant subspace are labelled by their eigenvalues: $m = -j,-j+1,\dots,j-1,j$.
  \end{theorem}
  \onslide<3->\textbf{Consequences:}
  \begin{itemize}
    \item<4-> One can obtain the explicit form of $\vec{J}$ and subsequently its components $J_x,J_y,J_z$. These are precisely the angular momentum operators in quantum mechanics.
    \item<5-> The eigenvalues of $\vec{J}^2$ and $J_z$ are $j(j+1)$ and $m$, respectively\footnote<5->{Typically, the $z$-axis is chosen as the standard axis.}. In quantum physics, these eigenvalues correspond to the observable total angular momentum and its $z$-component.
    \item<6-> This generalizes to other types of angular momentum, such as \textit{spin angular momentum}!
  \end{itemize}

  % \begin{itemize}
  %   \item Discuss connection between generators and quantum operators, eigenvalues and classical observables, discretization (!), etc.
  %   \item This is the kicker. I will get very excited here probably.
  % \end{itemize}

\end{frame}

%%%%%%%%%%%%%%%%%%%%%%%%%%%%%%%%%%%%%%%%%%%%%%%%%%%%%%%%%%%%%%%%%%%%%

\begin{frame}{Connection to Quantum Mechanics: Punchline}
  
  {
    \setbeamertemplate{blocks}[rounded][shadow=true]
    \setbeamercolor{block title}{bg=red!75!black, fg=white} % Customize title color
    \setbeamercolor{block body}{bg=red!10} % Customize body color
    \begin{block}<2->{Discretization of Angular Momentum for Free}
      Arguably the most defining characteristic of quantum mechanics is that classically measurable quantities become discretized (quantized) when observed on the quantum scale. Without any physical motivation, the irreducible representations of SO(3) gave it to us for free!
    \end{block}
  }

  \vfill

  \onslide<3->{\textit{Now is an appropriate time to let some tears out.}}

  \vfill
  
  \onslide<4->{But that's not all folks!}
  
    \vfill
  
\end{frame}

%%%%%%%%%%%%%%%%%%%%%%%%%%%%%%%%%%%%%%%%%%%%%%%%%%%%%%%%%%%%%%%%%%%%%

\begin{frame}[t]{Conservation of Angular Momentum}

  % \begin{definition}<1->
    \begin{enumerate}
      \item<1-> The \textcolor{CPgold}{\textit{commutator}} of two operators $A$ and $B$ is defined as $[A,B] = AB - BA$.
      \item<2-> The \textcolor{CPgold}{\textit{Hamiltonian}} operator $\hat{H}$ is the quantum mechanical operator corresponding to the total energy of a system.
    \end{enumerate}
  % \end{definition}

  \begin{theorem}<3->[Ehrenfest]
    If a time-independent Hermitian operator commutes with the Hamiltonian, then the physical observable corresponding to the operator is conserved.
  \end{theorem}

  % \onslide<4->{
  %   % \textbf{The Gist:}
  %   \begin{itemize}
  %     \item<4-> The differential forms of the angular momentum operators $J_x,J_y,J_z$ can be derived with some additional arguments.
  %     \item<5-> One can show that any system with radial symmetry is invariant under SO(3) rotations, and thus the angular momentum operators commute with the Hamiltonian.
  %   \end{itemize}
  % }

  {
    \setbeamertemplate{blocks}[rounded][shadow=true]
    \setbeamercolor{block title}{bg=red!75!black, fg=white} % Customize title color
    \setbeamercolor{block body}{bg=red!10} % Customize body color
    \begin{block}<4->{Consequences}
      \begin{enumerate}[\textcolor{red!25!black}{\bfseries\arabic{enumi}.}]
        \item<4-> Any system with radial symmetry is invariant under SO(3) rotations, so $[ \hat{H}, \vec{J} ] = 0$.
        \item<5-> Conservation of angular momentum is a direct result of the radial symmetry of the system.
        \item<6-> Similar arguments can be made for the continuous group of translations in space, leading to the conservation of linear momentum for translationally invariant systems.
      \end{enumerate}
    \end{block}
  }

  % \begin{itemize}
  %   \item Introduce the commutator.
  %   \item Do commutator example with Hamiltonian and $J$.
  %   \item Discuss implications.
  %   \item We can do the same thing for translation group which gives us the familiar $\hat{p}$ operator and conservation of linear momentum!
  % \end{itemize}
  
  \vfill

\end{frame}

%%%%%%%%%%%%%%%%%%%%%%%%%%%%%%%%%%%%%%%%%%%%%%%%%%%%%%%%%%%%%%%%%%%%%

\begin{frame}{Additional Applications}
  
  \begin{enumerate}
    \item<2-> The $j=1/2$ irreducible representation of SO(3) describes \textit{fermions}. A modified periodicity condition due to the half-integer representation leads to \textit{spinors}!
    \item<3-> We can take tensor products of the irreducibles of SO(3) to obtain multi-particle states, arriving at results such as:
    \begin{itemize}
      \item<3-> Clebsch-Gordan coefficients
      \item<4-> singlet versus triplet states
      \item<5-> the Pauli exclusion principle
    \end{itemize}
  \end{enumerate}
  
  \vfill
  
  \textit{This is the tip of the iceberg!}

  \vfill

\end{frame}
  
%%%%%%%%%%%%%%%%%%%%%%%%%%%%%%%%%%%%%%%%%%%%%%%%%%%%%%%%%%%%%%%%%%%%%


\section{The Braid Group}
\begin{frame}
  \sectionpage
\end{frame}

%%%%%%%%%%%%%%%%%%%%%%%%%%%%%%%%%%%%%%%%%%%%%%%%%%%%%%%%%%%%%%%%%%%%%

\begin{frame}{Basic Definitions}

  \vfill

  \begin{itemize}
    \item Formal definitions.
    \item Physical/intuitive visualization and interpretation.
    \item Standard generators.
    \item Automorphisms of $\pi_1(\D_n)$.
    \item Braid relations in this picture.
    \item 1D Reps.
    \item Burau representation.
    \item Note on faithfulness.
    \item Unitary representation from reduced Burau.
  \end{itemize}

  \vfill

\end{frame}

%%%%%%%%%%%%%%%%%%%%%%%%%%%%%%%%%%%%%%%%%%%%%%%%%%%%%%%%%%%%%%%%%%%%%


\section{Physical Applications of the Braid Group}
\begin{frame}
  \sectionpage
\end{frame}

%%%%%%%%%%%%%%%%%%%%%%%%%%%%%%%%%%%%%%%%%%%%%%%%%%%%%%%%%%%%%%%%%%%%%

\begin{frame}{Rotations of Quantum Hilbert Space}

  \vfill

  \begin{itemize}
    \item 1D action on Hilbert space, permuting particles, compare/contrast to bosons/fermions.
    \item Talk about nontrivial braiding effects.
    \item Example of unitary braid rep acting on Hilbert space.
  \end{itemize}

  \vfill

\end{frame}

%%%%%%%%%%%%%%%%%%%%%%%%%%%%%%%%%%%%%%%%%%%%%%%%%%%%%%%%%%%%%%%%%%%%%

\begin{frame}{Anyons: A Consequence of Braiding}

  \begin{itemize}
    \item Introduce anyons.
    \item Discuss how anyons are described by the braid group.
    \item Fusion rules, abelian vs nonabelian anyons.
    \item Non-interacting anyons.
    \item Non-interacting anyons in harmonic potential.
    \item Nontrivial braiding effects anyone?
    \item Applications of anyons! (quantum computing, topological quantum field theory, FQHE, etc.)
  \end{itemize}

\end{frame}

%%%%%%%%%%%%%%%%%%%%%%%%%%%%%%%%%%%%%%%%%%%%%%%%%%%%%%%%%%%%%%%%%%%%%

\begin{frame}{Summary/Conclusion}

  Acknowledgements, questions, references (?)

\end{frame}

%%%%%%%%%%%%%%%%%%%%%%%%%%%%%%%%%%%%%%%%%%%%%%%%%%%%%%%%%%%%%%%%%%%%%
\end{document}