\chapter{Introduction}\label{ch:introduction}

The intimate connection between abstract mathematics and the physical world highlights the beautiful complexity of nature. In our efforts to understand the fundamental processes that govern the universe, we grow increasingly reliant on the principles of mathematics. One such tool that bridges the abstract and physical regimes is known as representation theory.

In this thesis, we examine representation theory and observe the physical consequences that emerge from the mathematics. First, we begin in \cref{ch:rep_background} with a brief overview of representation theory. Then in \cref{ch:Phys_applications}, we elucidate specific applications of representation theory in the context of quantum physics, arriving at remarkably fundamental results. In \cref{ch:braid_group}, we introduce the braid group and explore some of its representations. Finally, we highlight physical applications of the braid group and its representations in \cref{ch:anyons}.

It is assumed that the reader has a basic understanding of group theory and linear algebra. A familiarity with topology is also required, and the relevant definitions can be found in \cref{ch:top_background}. Additionally, notational conventions are chosen to align with the physics literature which at times differs from the standard mathematical notation. Some discussions in the later chapters also assume a general understanding of classical and quantum mechanics. To those less familiar with those topics and the notation, a supplementary overview is given in \cref{ch:physics_background}.