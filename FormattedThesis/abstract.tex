Representation theory, which encodes the elements of a group as linear operators on a vector space, has far-reaching implications in physics.
Fundamental results in quantum physics emerge directly from the representations describing physical symmetries. We first examine the connections between specific representations and the principles of quantum mechanics. Then, we shift our focus to the braid group, which describes the algebraic structure of braids. We apply representations of the braid group to physical systems in order to investigate quasiparticles known as anyons. Finally, we obtain governing equations of anyonic systems to highlight the physical differences between braiding statistics and Bose-Einstein/Fermi-Dirac statistics.